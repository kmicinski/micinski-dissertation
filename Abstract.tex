%Abstract Page 

\hbox{\ }

\renewcommand{\baselinestretch}{1}
\small \normalsize

\begin{center}
\large{{ABSTRACT}} 

\vspace{3em} 

\end{center}
\hspace{-.15in}
\begin{tabular}{ll}
Title of dissertation:    & {\large  INTERACTION-BASED PRIVACY POLICIES}\\
&				      {\large  FOR MOBILE APPS} \\
\ \\
&                          {\large  Kristopher Micinski, Doctor of Philosophy, 2017} \\
\ \\
Dissertation directed by: & {\large  Professor Jeffrey S. Foster} \\
\end{tabular}

\vspace{3em}

\renewcommand{\baselinestretch}{2}
\large \normalsize

Mobile operating systems pervade our modern lives. Security and
privacy is of particular concern on these systems, as they have access
to a wide range of sensitive resources. Apps access these sensitive
resources to help users perform tasks. However, apps may use these
sensitive resources in a way that the user does not expect. For
example, an app may look up reviews of restaurants nearby, but also
leak the user's location to an ad service every hour.

I claim that interaction serves as a valuable component of security
decisions, because the user's interaction with the app's user
interface (UI) deeply informs their mental model of what is
acceptable. I introduce the notion of interaction-based security,
wherein security decisions are driven by this interaction.

To help understand and enforce interaction-based security, I present
four pieces of work. The first is Redexer, which allows performing
binary instrumentation of off-the-shelf Android binaries. Binary
transformation is a useful tool for enforcing and studying security
properties. I demonstrate one example of how Redexer can be used to
study location privacy in apps.

Android permissions constrain how data enters apps, but does not
constrain how the information is used or where it
goes. Information-flow allows us to formally define what it means for
data to leak from applications, but it is unclear how to use
information-flow policies for Android apps, because apps frequently
declassify information. My insight is that declassification should be
driven by the user's interaction with the app's UI. I
define interaction-based declassification policies, and show how they
can be used to define policies for several example apps. I then
implement a symbolic executor which checks Android apps to ensure they
respect these policies.

Next, I test the hypothesis that the app's UI influences security
decisions. I outline an app study that measures when apps use
sensitive resources with respect to their UI. I then conduct a user
study to measure how an app's UI influences their expectation that a
sensitive resource will be accessed. I find that interactivity plays a
large role in determining user expectation of sensitive resource
use. I also find that users may not always understand background uses
of these sensitive resources and using them expectation requires
special care in some circumstances.

Last, I present a tool which can help a security auditor quickly
understand how apps use resources. My tool uses a novel combination of
app logging, symbolic execution, and abstract interpretation to infer
a formula that holds on each permission use. I evaluate my tool on
several moderately-sized apps and show that it infers the same
formulas we laboriously found by hand.

