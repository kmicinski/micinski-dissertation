\section{Related work}

Krumm \cite{krumm:loc_priv_survey} describes 

Shokri \cite{shokri:quantifying_loc_privacy} notes the lack of a
quantifiable metric to reason about location privacy.  They present a
formalism based on an attacker's prior knowledge and present a
systematic bound on how much the attacker could have learned given a
data trace, implementing their technique in a tool (the location
privacy meter) which will inform users, over time, of how much privacy
they have lost in using a system.

http://icapeople.epfl.ch/rshokri/papers/11SP.pdf

Benisch et. al \cite{benisch:capturing_location_priv_prefs} performed
a user study on a set of test subjects and acquired data as to the
preferences of users' privacy preferences.  They revealed that users
do indeed have rich preferences as to whom they share their data with,
and present a systematic approach to deriving policies for sharing
location data with other users based on a set of rules.

http://www.springerlink.com/content/q5028286x0n03w25/

Cheng et. al \cite{cheng:preserving_user_location_privacy} studies a
similar location dissemination approach to the one considered here:
where location information is disclosed to services which are used to
provide information to users based on their location.  They suggest
using a data model which incorporates uncertainty, in which case
queries to services embed probabilistic results (with respect to
quality) and present an approach to minimize path projection.

http://www.petworkshop.org/2006/preproc/preproc_23.pdf

The Cach\'{e} system uses caching of query results along with
prefetching to help ensure user anonymity.  An application writer will
specify a set of triggers which will be passed down to the cach\'{e}
system.  The Cach\'{e} system mediates requests and prefetches data as
appropriate.  The risk is that the user may not have the data in their
cache (or it may be stale), and to account for this the system does an
amount of learning and programmer specification to maximize
probability of fresh results residing in the cache.

http://www.winlab.rutgers.edu/~janne/cache-mobisys11.pdf

Skimmed this paper on

http://dl.acm.org/citation.cfm?id=1128483

Survey paper on location privacy, I need to read about mix zones

http://www.utdallas.edu/~muratk/courses/privacy08f_files/location_privacy_pervasive_computing.pdf

http://citeseerx.ist.psu.edu/viewdoc/summary?doi=10.1.1.1.3689