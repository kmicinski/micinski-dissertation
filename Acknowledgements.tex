%Acknowledgments

\renewcommand{\baselinestretch}{2}
\small\normalsize
\hbox{\ }
 
\vspace{-.65in}

\begin{center}
\large{Acknowledgments} 
\end{center} 

\vspace{1ex}

I would not have completed this dissertation without the help of many
people who offered thoughtful and challenging advice and perspective.

My advisor, Jeff Foster, was instrumental in helping develop an
appreciation for good science. I was initially surprised at Jeff's
outlook on research. Instead of being focused on a particular
collection of techniques or research areas, Jeff was someone who cared
foremost about problems.  He inspired a sense that there will always
be exciting research problems to work on---especially if we're willing
to step outside of the area with which we're comfortable.

Michelle Mazurek also provided deeply helpful direction. Michelle
taught me an appreciation for techniques from Human-Computer
Interaction, and was crucial in helping form the parts of my thesis
related to usability. Although not a programming language researcher,
Michelle quite capably understood and thoughtfully commented on my
work in that area.

I was also guided by several other professors at Maryland and
elsewhere. Michael Hicks offered lots of useful advice, and provided a
different (but equally thoughtful) perspective from Jeff. He has
consistently challenged me to acheive more. David Van Horn was helpful
in providing some interesting discussions on static analysis and his
methods always inspired my thought. Last, Michael Clarkson played a
formative role in my thinking about formalizations of security. My
thesis was deeply informed by our work together on temporal logics for
hyperproperties. He inspired me as someone who was both a great
teacher and researcher, and I hope to teach as well as he someday.

The other members of the PLUM lab were also great friends and provided
useful feedback and direction. Jinseong Jeon was incredibly kind but
fiercely intelligent, and mentored me during my first years. Jinseong
could write enormous amounts of code acting as if it was no big
deal. He answered many hasty emails near deadlines. David Darais and I
have had many interesting conversations about both programming
languages and coffee. He has urged me to pursue ideas I might have
given up on, and I look forward to his continued input. Tom Gilray
shares many of my goals on visualization and static analysis, and we
come at programming languages from a similar perspective. He has
helped me understand the nuts and bolts of abstract interpretation in
a way that distilled it simply. He has been kind but critical in
critiquing my ideas.

Daniel Votipka began his PhD during my last year, and performed at the
level of a very advanced graduate student quickly. I was impressed and
thankful at how much he was able to get done while having relatively
little experience in the areas we were working in. Rock Stevens was
also very useful, and a master scripter. He had a laser focus on
achieving goals, and I have no idea how he was able to download a few
hundred apps from Google Play in a few hours, circumventing the
security controls which I assumed would stop us. I was fortunate
enough to work with a number of talented undergrads, including Philip
Phelps and Rebecca Norton.

Many, many people reviewed my papers over the years and offered useful
and constructive feedback. Among these are Sascha Fahl, Yasemin Acar,
and members of HCIL at Maryland.

This research was supported in part by NSF CNS-1064997 and by a
research award from Google. The work was also supported by the
partnership between UMIACS and the Laboratory for Telecommunication
Sciences.

Last, I am greatful to my family and friends. My parents have been
incredibly supportive of all of my goals, in every way I might
imagine. I owe them more than I can give. They gave me a lot of life
advice that helped me succeed in grad school, and nurtured an
appreciation for honesty and self-accountability.


