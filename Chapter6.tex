\renewcommand{\thechapter}{6}

\chapter{Conclusion and Future Directions}

In this dissertation, I defined interaction-based security as security
which was informed by the user's interaction with an app's UI. I then
presented several pieces of work showing how we could move the space
of mobile apps closer towards accomodating interaction-based
security. In this chapter, I will briefly review each piece of work
and discuss how it fits in to the larger space of mobile apps and
systems going forward, and detail some future directions of work.

\section{Binary Rewriting for Android Apps}

Much of the work in my dissertation relied on the ability to perform
binary instrumentation of Android apps. Binary instrumentation is a
powerful mechanism and allows us to perform a range of dynamic
analyses, only some of which have been presented here. In Chapter 2
detailed some of the key issues involved in transforming Android apps
correctly. I describe Redexer---my binary rewriter for Android---and
its implementation, along with a demonstration of how Redexer be used
for studying location truncation in Android apps.

Redexer currently works at scale on arbitrarily large Android
apps. During my work for Chapter 4, we were using Redexer on most top
apps from Google Play. The logging framework was pushed even further
during my work on Chapter 5, wherein we logged every method invocation
in several production-sized apps. Generally, Redexer offers a fairly
flexible API for app rewriting and could be used for applications not
described here. For example, we are currently working on using Redexer
to understand which data apps back up. I also envision using Redexer
to enforce the security policies that we understand via the work in
Chapter 5. For example, this might happen by using Hogarth to figure
out the policy for an app and then using a dynamic enforcement
mechanism in Redexer to exit the app if the app goes down a path not
observed in the set of analyzed runs.

\section{Checking Interaction-Based Declassification Policies for Android Using Symbolic Execution}

While Redexer allows rewriting apps to change their dynamic behavior,
Chapter 3 explores the complications of understanding information-flow
in apps. Information flow is a tricky property, because it relies on
quantifying over sets of runs, rather than a single run. My work on
interaction-based information flow policies was driven by an intuition
that users reason about declassification decisions using the app's UI. 

I did not mention information flow in subsequent chapters, and instead
focused on permission uses. I would like to go back and study user
preferences and security expectations in the presence of information
flow. As we saw in Chapter 3, information-flow is still challenging to
apply to production apps, compared to, e.g., permission systems. I
plan to study what has to be done to support information flow for
production apps and if it would even be beneficial to users (versus
better permission systems).

\section{User Interactions and Permission Use on Android}

Chapter 4 details my work in understanding the relation between user
interactions and permission in Android apps. We found that the Android
system is largely heading in the right direction by urging apps to
authorize permission use on first use. However, we found that users
may not fully understand background permission uses, specifically when
they occur in apps which also have foreground uses of those
permissions.

I plan to continue this work trying to understand how we could better
design UIs to help users understand background permission uses. This
was challenging in AppTracer, because AppTracer was driven by a
temporal notion of influence: it was hard to reason about why a
permission use happened when its cause was far removed from the UI
event that caused it. This motivated me to build Hogarth, which allows
understanding this dependence in a principled way by registrations of
callbacks. However, I have not applied Hogarth to large apps of the
variety in Chapter 4. I plan to continue work on tools like Hogarth,
so that I can apply them to production apps and understand why apps
use background resources. I hope this will help inform my knowledge of
why these uses are necessary and what we can do to explain them to
users.

\section{Permission-Use Provenance in Android Using Sparse Dynamic Analysis}

Finally, Chapter 5 introduces Hogarth, a system which uses app logs as
the basis for program understanding. Understanding interaction-based
security requires very precise knowledge about paths through an
app. Traditional types of program analyses are challenging to apply
here because they achieve scalability by sacrificing
precision. Hogarth provides a very precise understanding of why a
permission is used in an app, but sacrifices soundness to do so.

There are many exciting directions to explore following up
Hogarth. First, the initial implementation of Hogarth needs to be
scaled up to work on realistically sized apps. I have a number of
ideas for how to do so. For example, one limiting factor in Chapter 5
was that Hogarth ran out of memory processing very large logs. This
was because a symbolic state was created for each entry in the
log. Instead, I will modify Hogarth so that it performs on-the-fly
minimization, merging newly discovered symbolic states as it
interpolates the log rather than generating all of the symbolic states
before minimization.

Next, I would like to understand how to make fundamental improvements
to Hogarth. For example, Hogarth will frequently include in its path
conditions the post conditions for each loop through which it
passes. Although this is the correct behavior, I have frequently found
that this information is not usually necessary. Also, Hogarth uses
only positive examples as it performs its minimization. I speculate
that Hogarth may be able to present a more minimal path condition if
it were to consider the negative paths (which did not lead to a
permission use) and prune information from the path condition which
also occurred in those paths.

More generally, I believe Hogarth is just a first piece of work
towards using program analysis to aid reverse engineering. I am
uncertain what real reverse engineers would think of Hogarth. I assume
that we would discover many interesting problems if we were to give
Hogarth to professional reverse engineers. I am very much interested
in doing so, and using other techniques from program analysis to help
inspire advances in program understanding.

