\paragraph{Revealing contacts in a designated group}

Imagine an application that interacts with a remote server that allows
a user to backup selected contacts so that they may later be
retrieved.  The set of the user's contacts is private information, and
the app will query the user to pick a contact to send to the service
so that it may be backed up.  The property we wish to maintain is that
the server can never learn the value of a contact until it has been
selected from the list.

The following is a correct implementation of this functionality:

\input{examples/share_private_contacts.pi}

\input{examples/share_private_contacts.policy}

\paragraph{Selecting a private contact from a list}

In this example, the user selects a private contact from a contact
picker provided to the user.  The idea is that the attacker is allowed
to learn the selected contact, but no more information about the list.

This is an example of the core formalism:
\begin{itemize}

\item The user sends the private data (a \code{string list}) to the
  application
\item The user selects a number within the range $0$ to $length(x) -
  1$ to index into the list
\item The application sends only \code{x[y]} to the remote application
  through the special channel \code{net}.
\item The remote application reads the selected contact through the
  \code{net} channel.
\end{itemize}

\input{examples/private_list_select.pi}

We next want to move from this core formalism to a formalism where we
actually lay out GUI buttons and events.  Specifically, we want a
formalism that will give us a spinner, and a button to select from a
set of possible outputs.  We want the action to be such that when the
user from a list, it is sent to the web:

\input{examples/private_list_select_gui.pi}

Now we need to specify the policy on $x$:

\input{examples/private_list_select_gui.policy}

\paragraph{Toggling between city and state}

In the following example, we continually send our location to a remote
server (for example, for location based search results).  We want the
GUI to have a checkbox that can be in two states:

\begin{itemize}
\item If the checkbox is checked, we send fine grained location
  information to the server.
\item If the checkbox is unchecked, we mask our location so that we
  only reveal our location to within a mask.
\end{itemize}

The following program implements this functionality:

\input{examples/toggle_location_granularity.pi}

The security policy is:

\input{examples/toggle_location_granularity.policy}
