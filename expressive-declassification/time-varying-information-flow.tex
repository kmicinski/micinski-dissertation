\documentclass{article}
\usepackage{graphicx}
\usepackage{amsmath}
\usepackage{amsfonts}
\usepackage{amssymb}
\usepackage{listings}

\newcommand{\aset}[1]{\{#1\}}
\newcommand{\dom}{\mathop\textit{dom}\nolimits}

\newcommand{\sfmt}[1]{\textsf{#1}}
\newcommand{\sch}{\textit{ch}}
\newcommand{\loc}{\ell}
\newcommand{\sassign}[2]{#1 := #2}
\newcommand{\scase}[2]{\sfmt{case}~#1~\sfmt{of}~#2}
\newcommand{\sderef}[1]{!#1}
\newcommand{\sfalse}{\sfmt{false}}
\newcommand{\sif}[3]{\sfmt{if}~#1~\sfmt{then}~#2~\sfmt{else}~#3}
\newcommand{\sinl}{\sfmt{inl}}
\newcommand{\sinr}{\sfmt{inr}}
\newcommand{\sinstall}[2]{\sfmt{install}~#1~#2}
\newcommand{\sdeclassify}[1]{\sfmt{declassify}~#1}
\newcommand{\sref}[1]{\sfmt{ref}~#1}
\newcommand{\denot}[1]{\ensuremath{\llbracket #1 \rrbacket}}
\newcommand{\ssend}[2]{\sfmt{send}~#1~#2}
\newcommand{\strue}{\sfmt{true}}
\newcommand{\sunit}{\sfmt{unit}}
\newcommand{\sreduce}{\Downarrow}
\newcommand{\treduce}{\rightarrow}
\newcommand{\partialfun}{\rightharpoonup}
\newcommand{\ltrue}{\ensuremath{\top}}
\newcommand{\lfalse}{\ensuremath{\bot}}
\newcommand{\config}[1]{\langle{}#1\rangle{}}
\newcommand{\program}[1]{\ensuremath{\mathcal{#1}}}
\newcommand{\restr}[2]{\ensuremath{\left.#1\right|_{#2}}}
\newcommand{\modelsrcon}{\models^{r}}
\newcommand{\judge}{\vdash}
\newcommand{\xv}{p}

\newcommand{\tr}{t}
\newcommand{\typ}{\tau}
\newcommand{\tint}{\textit{int}}
\newcommand{\tset}{\mathcal{T}}
\newcommand{\pset}{\ensuremath{\mathcal{P}}}
\newcommand{\tnext}{\mathcal{X}}
\newcommand{\talways}{\square}
\newcommand{\tevent}{\lozenge}
\newcommand{\tuntil}{~\mathcal{U}~}
\newcommand{\tsince}{~\mathcal{S}~}
\newcommand{\tpast}{~\mathcal{P}~}
\newcommand{\tknows}[1]{\mathcal{K}_{#1}}
\newcommand{\tatmost}[1]{\mathcal{N}_{#1}}
\newcommand{\tpossible}[1]{\mathcal{L}_{#1}}
\newcommand{\tthen}{~\textit{then}~}
\newcommand{\tlast}[2]{\textit{last}(#1, #2)}
%\newcommand{\tforall}{\hbox{forall}~}
\newcommand{\texists}{\hbox{exists}~}
\newcommand{\tcurrent}[1]{\hbox{current}_{#1}~}
\newcommand{\tval}[1]{\textit{value}(#1)}
\newcommand{\trelease}{\rhd}
\newcommand{\evt}{\eta}

\begin{document}

\title{About time varying information flow for reactive systems}
\maketitle

\section{Introduction}

Assume a trace generating semantics for a program $e$ that takes the
form $e \Downarrow tr$, where $tr$ is a sequence of messages of the
form $ch \{!,?\} v$ for a channel type $\mathbb{C}$ (identifiers) that
can be compared for equality, but are fixed (i.e., cannot be freely
generated) and $v$ is a set of values that appears along these
channels.  We use $!$ for message output along channel $ch$ and $?$
for input along channel $ch$.

The semantics of the program generates a trace of message by taking
reduction steps in an abstract machine given a small step semantics of
the form $\Sigma_1 \to^\eta \Sigma_2$, where $\eta$ is either a
message or an empty transition ($\tau$).

By a principal $p$ we mean a subset of $\mathcal{P}(\mathbb{C})$.  The
interpretation of a principal is that when the semantics of a program
produces output, the principal will observe only the messages in the
set $p$.  By observing the messages written along the channels in $p$,
they will potentially learn values along other channels.  For example,
the program (1):

\begin{lstlisting}[float,caption=My caption here,label=ex1]
input(secret).output<secret>
\end{lstlisting}

When run produces a trace that contains:

\begin{verbatim}
secret?n , output!n
\end{verbatim}

If we change the value of the input, we change the value of the
output.  An intelligent observer will learn that the value input along
the channel secret has been revealed.

Contrast with the following program:

\begin{lstlisting}[float,caption=My caption here,label=ex2]
input(secret).output<secret*0>
\end{lstlisting}

Which always produces the observable 0.

\section{Time varying information flow} 

Now we want to define information flow polices for programs that
interact with secret inputs whose values change over time, along with
the policies that change them over time.

\paragraph{Noninterference}

Let $Tr(e)$ be the set of traces of program $e$.  We define
noninterference for channel $ch$ at time $i$ in the following way:

Consider any trace $tr \in Tr(e)$, if $tr[i] = ch?v_1$ for some $v_1$,
then there is also a trace in $Tr(e)$ which is equivalent up to $i$,
but contains any other value $v_2$, call this set $\equiv(tr,i-1)$,
the set of traces equivalent up to $i-1$ steps.  Let $succ(tr,i)$ be
the suffixes of $tr$ after $i$ steps.  $tr$ satisfies noninterference
at time $i$ if for all $tr_1 tr_2 \in \equiv(tr,i-1)$, $succ(tr_1,i) =
succ(tr_2,i)$.

A program satisfies time varying noninterference for channel $ch$ if
for all times $i$ it satisfies noninterference at time $i$.

\paragraph{Defining knowledge about a value over time}

Given a channel $ch$ and an index $i$, we define the knowledge about
$ch$ a time $j \geq i$ as the set of literals that is known about the
value of $ch[i]$ at time $j$.

As an example, in Program \ref{ex1} above, we have the following
knowledge graph for secret at time 0:

time 0: $\{\}$ (No information known)
time 1: $value(secret=0)$ 

By contrast in program \ref{ex2} we have the knowledge graph as zero
throughout.  Note that the knowledge for times before i is undefined.
We typically want to write formulas that have the form: ``considering
the value of the secret at the current time, it is never true that any
time in the future, the observer learns more than $\phi(secret)$.''

\section{An epistemic semantics for time varying data}


\begin{figure*}[t]
  \begin{displaymath}
    \begin{array}{rcl}
      % \phi^\mathcal{T} & ::= & x \mid p^\mathcal{T} \mid f^\mathcal{T}
      % ( \phi^{\mathcal{T}}_1 , ... , \phi^\mathcal{T}_n ) \mid
      % \\
      \phi, \psi & ::= &
      \phi^A ( p_1 , ..., p_n ) @ x^t
      \mid \neg \phi
      \mid \phi \wedge \phi
      \mid \phi \vee \phi
      \mid \phi \tuntil \phi
      \mid \phi \tsince \phi \\
      & \mid & \talways \phi
      \mid \tevent \phi
      \mid \forall x . \phi
      \mid \tknows{\sch} \phi
      \mid Now(\phi,x)
    \end{array}
  \end{displaymath}

  \begin{displaymath}
    \begin{array}{rcl}
      \tset, \tr, i , S \models \phi^A ( p_1 , ..., p_n ) @ i & \iff &
      \tr[i] \models^A \phi^A ( p_1 , ..., p_n )  \\

      \tset, \tr, i , S \models \neg \phi & \iff &
      \tset, \tr, i , S \not \models \phi \\

      \tset, \tr, i , S \models \phi \land \psi & \iff &
      \tset, \tr, i , S \models \phi \hbox{ and } \tset, \tr, i , S \models \psi \\

      \tset, \tr, i , S \models \phi \lor \psi & \iff &
      \tset, \tr, i , S \models \phi \hbox{ or } \tset, \tr, i , S \models \psi \\

%      \tset, \tr, i \models \tnext \phi & \iff &
%      \tset, \tr, (i+1) \models \phi \\

      \tset, \tr, i , S \models \talways \phi & \iff &
      \forall j ~.~ j \geq i \Rightarrow (\tset, \tr, j , S \models \phi) \\

      \tset, \tr, i , S \models \tevent \phi & \iff &
      \exists j ~.~ j \geq i \Rightarrow (\tset, \tr, j , S \models \phi) \\

      \tset, \tr, i , S \models \phi \tuntil \psi & \iff &
      \exists j ~.~ j \geq i \Rightarrow ((\tset, \tr, j , S \models \psi)
      \hbox{ and } \\
      & & \forall k ~.~ i \leq k < j \Rightarrow (\tset, \tr, k , S
      \models \phi)) \\

      \tset, \tr, i , S \models \phi \tsince \psi & \iff &
      \exists j ~.~ j \leq i \Rightarrow ((\tset, \tr, j , S \models \psi)
      \hbox{ and } \\
      & & \forall k ~.~ j < k \leq i \Rightarrow (\tset, \tr, k , S
      \models \phi)) \\

      \tset, \tr, i , S \models \forall x . \phi & \iff &
      \hbox{for all } p, (\tset, \tr, i , S \models \aset{x\mapsto p} \phi) \\

      \tset, \tr, i , S \models \exists x , \phi & \iff &
      \hbox{there exists } p. (\tset, \tr, i , S \models \aset{x\mapsto p} \phi) \\

      \tset, \tr, i , S \models \tknows{\sch} \phi & \iff &
      \forall u \in \tset ~,~
      \tr[1..i] \cong^{\sch} u[1..i] \Rightarrow \\ & & 
        \tset, u, i , S \models
      \phi \\

      \tset, \tr, i , S \models Now(\phi,t) & \iff &
       \tset, \tr, i , S \models \phi \aset{t\mapsto i} \\

    \end{array}
  \end{displaymath}
  \label{fig:etl}
  \caption{Epistemic temporal logic semantics and notation.}
\end{figure*}

\begin{displaymath}
  \forall x , G(Now(L(secret@t = x),t))
\end{displaymath}

Or...

\begin{displaymath}
  \forall x , G(Now(\lnot F K(secret@t = x),t))
\end{displaymath}

Release properties...

\begin{displaymath}
  G (last\_checkbox\_true \Rightarrow Now(\forall x, G(secret@t = x)),t)
\end{displaymath}

\end{document}





