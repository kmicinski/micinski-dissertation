\documentclass{entcs} \usepackage{entcsmacro}
\usepackage{url}
\usepackage{latexsym,amsmath,amssymb,stmaryrd}

%\renewcommand{\bfdefault}{b}

\usepackage{xspace}
\usepackage{enumerate}
\usepackage{subcaption}
\usepackage{listings}
%\usepackage{supertabular}
%\usepackage{stackrel}
%\usepackage{marvosym}
\usepackage{pgf}
%\usepackage{tikz}
%\usepackage{mdframed}
\usepackage{mathpartir}
%\usepackage{etoolbox}
\usepackage{color}
\usepackage{soul}
\usepackage{subscript}
%\usepackage{float}

% Listings
\newcommand{\code}[1]{\text{\lstinline!#1!}}

\definecolor{lightgreen}{rgb}{.85,.95,.85}
\definecolor{lightblue}{rgb}{.85,.90,1}
\definecolor{lightred}{rgb}{.95,.85,.85}
\definecolor{lightgrey}{rgb}{.95,.95,.95}
\sethlcolor{lightred}

\definecolor{drkyellow}{rgb}{0.4,0.3,0}
\definecolor{drkred}{rgb}{0.5,0,0}
\definecolor{drkgreen}{rgb}{0,0.5,0}

\definecolor{dkgreen}{rgb}{0.1,0.40,0.1}

\definecolor{dkred}{rgb}{0.5,0,0}
\definecolor{medred}{rgb}{0.6,0,0}
\definecolor{gray}{rgb}{0.5,0.5,0.5}
\definecolor{mauve}{rgb}{0.58,0,0.82}

\newcommand{\myparagraph}[1]{\paragraph{#1.}}

\newcommand{\lstrule}{%
 \hspace{1mm}%
 \color{gray}{\rule[1.18mm]{0.8\textwidth}{0.1pt}}%
 \hspace{\fill}%
}

%% TODO: Make the listings look prettier

\lstset{language=Java,
  morecomment=[l]{--},          % add Haskell comment style
  columns=flexible,
  basicstyle=\small\sffamily,
%  basewidth=0.5em,
%  lineskip={-2.35pt},
%  aboveskip=5mm,
%  belowskip=-5mm,
%  identifierstyle=\textbf,
%  keywordsprefix=\#,
  literate={->}{$\rightarrow$}2
           {=>}{$\Rightarrow$}2
           {<-}{$\leftarrow$}2
           {...}{$\cdots$}2
           {fun}{$\lambda$}1
           {||}{$\parallel$}1
           {**}{$\times$}2,
  escapechar=@,
  escapeinside={/**}{*/},
  numbers=left,                   % where to put the line-numbers
  numberstyle=\tiny\color{gray},  % the style that is used for the line-numbers
  stepnumber=1,                   % the step between two line-numbers. If it's 1, each line 
                                  % will be numbered
  numbersep=5pt,                  % how far the line-numbers are from the code
  showspaces=false,               % show spaces adding particular underscores
  showstringspaces=false,         % underline spaces within strings
  numberblanklines=false,
  showtabs=false,
  frame=single,                   % adds a frame around the code
%  frame=lines,
  rulecolor=\color{gray},        % if not set, the frame-color may be changed on line-breaks within not-black text (e.g. commens (green here))
%  tabsize=2,                      % sets default tabsize to 2 spaces
%  captionpos=b,                   % sets the caption-position to bottom
%  breaklines=false,                % sets automatic line breaking
%  breakatwhitespace=false,        % sets if automatic breaks should only happen at whitespace
%  title=\lstname,                   % show the filename of files included with \lstinputlisting;
%  keywordstyle=\color{blue},          % keyword style
%  identifierstyle=\color{drkgreen},     
%  backgroundcolor=\color{lightgrey},
  commentstyle=\color{dkred}\itshape,       % comment style
  stringstyle=\color{mauve}         % string literal style
}

% \lstset{language=Caml,
%   morecomment=[l]{--},          % add Haskell comment style
%   columns=flexible,
%   basicstyle=\small\sffamily,
% %  basewidth=0.5em,
% %  lineskip={-2.35pt},
% %  aboveskip=5mm,
% %  belowskip=-5mm,
% %  identifierstyle=\textbf,
% %  keywordsprefix=\#,
%   literate={->}{$\rightarrow$}2
%            {=>}{$\Rightarrow$}2
%            {<-}{$\leftarrow$}2
%            {..}{$\cdots$}2
%            {fun}{$\lambda$}1
%            {||}{$\parallel$}1
%            {**}{$\times$}2,
%   escapechar=@,
%   escapeinside={/**}{*/},
%   numbers=left,                   % where to put the line-numbers
%   numberstyle=\tiny\color{gray},  % the style that is used for the line-numbers
%   stepnumber=1,                   % the step between two line-numbers. If it's 1, each line 
%                                   % will be numbered
%   numbersep=5pt,                  % how far the line-numbers are from the code
%   showspaces=false,               % show spaces adding particular underscores
%   showstringspaces=false,         % underline spaces within strings
%   numberblanklines=false,
%   showtabs=false,
%   frame=single,                   % adds a frame around the code
% %  frame=lines,
%   rulecolor=\color{gray},        % if not set, the frame-color may be changed on line-breaks within not-black text (e.g. commens (green here))
% %  tabsize=2,                      % sets default tabsize to 2 spaces
% %  captionpos=b,                   % sets the caption-position to bottom
% %  breaklines=false,                % sets automatic line breaking
% %  breakatwhitespace=false,        % sets if automatic breaks should only happen at whitespace
% %  title=\lstname,                   % show the filename of files included with \lstinputlisting;
% %  keywordstyle=\color{blue},          % keyword style
% %  identifierstyle=\color{drkgreen},     
% %  backgroundcolor=\color{lightgrey},
%   commentstyle=\color{dkred}\itshape,       % comment style
%   stringstyle=\color{mauve}         % string literal style
% }


%\lstset{emph={},emphstyle={}}%
%\lstset{morekeywords={\!, (, \), \., \<, \>, ref, in, if,then,else,let, install, <,>},keywordstyle={\color{blue}\bfseries}}%



%% Theorem stuff
%\theoremstyle{definition}
%\newtheorem{algorithm}{Algorithm}
%\newtheorem{definition}{Definition}

\newcommand{\aset}[1]{\{#1\}}
\newcommand{\dom}{\mathop\textit{dom}\nolimits}

%% Names of things
\newcommand{\tvetl}{TVETL\xspace}

\newcommand{\sfmt}[1]{\textsf{#1}}
\newcommand{\sch}{\textit{ch}}
\newcommand{\loc}{\ell}
\newcommand{\sassign}[2]{#1 := #2}
\newcommand{\scase}[2]{\sfmt{case}~#1~\sfmt{of}~#2}
\newcommand{\sderef}[1]{!#1}
\newcommand{\sfalse}{\sfmt{false}}
\newcommand{\sif}[3]{\sfmt{if}~#1~\sfmt{then}~#2~\sfmt{else}~#3}
\newcommand{\bdenot}[1]{\llbracket~#1\rrbracket_B}
\newcommand{\sinl}{\sfmt{inl}}
\newcommand{\sinr}{\sfmt{inr}}
\newcommand{\sinstall}[2]{\sfmt{install}~#1~#2}
\newcommand{\sdeclassify}[1]{\sfmt{declassify}~#1}
\newcommand{\sref}[1]{\sfmt{ref}~#1}
\newcommand{\denot}[1]{\ensuremath{\llbracket #1 \rrbacket}}
\newcommand{\ssend}[2]{\sfmt{send}~#1~#2}
\newcommand{\strue}{\sfmt{true}}
\newcommand{\sunit}{\sfmt{unit}}
\newcommand{\sreduce}{\Downarrow}
\newcommand{\treduce}{\rightarrow}
\newcommand{\partialfun}{\rightharpoonup}
\newcommand{\ltrue}{\ensuremath{\top}}
\newcommand{\lfalse}{\ensuremath{\bot}}
\newcommand{\config}[1]{\langle{}#1\rangle{}}
\newcommand{\program}[1]{\ensuremath{\mathcal{#1}}}
\newcommand{\sclass}[2]{\ensuremath{\lfloor#1\rfloor_{#2}}}
\newcommand{\restr}[2]{\ensuremath{\left.#1\right|_{#2}}}
\newcommand{\modelsrcon}{\models^{r}}
\newcommand{\amodels}{\models^{A}}
\newcommand{\judge}{\vdash}
\newcommand{\xv}{p}
\newcommand{\bexpr}{b}
\newcommand{\filter}{\upharpoonright}
\newcommand{\restrict}{\upharpoonright}
\newcommand{\etlb}{ETLB\xspace}
\newcommand{\hyperltltwo}{HyperLTL\xspace}

\newcommand{\absstate}{\Sigma^s}
\newcommand{\tvar}{x^t}

\newcommand{\traces}{\textit{traces}}
\newcommand{\tracesrename}{\textit{traces'}}
\newcommand{\tick}[1]{#1^{+}}
\newcommand{\atom}{A}
\newcommand{\toolname}{AGIF\xspace}

\newcommand{\tr}{t\xspace}
\newcommand{\typ}{\tau}
\newcommand{\tint}{\textit{int}}
\newcommand{\tset}{\ensuremath{\mathcal{T}}\xspace}
\newcommand{\tsets}{\ensuremath{\mathcal{T}^s}\xspace}
\newcommand{\pset}{\ensuremath{\mathcal{P}}}
\newcommand{\tnext}{\mathcal{X}}
\newcommand{\talways}{\mathcal{G}}
%\newcommand{\tfuture}{~\mathcal{F}~}
\newcommand{\tevent}{\lozenge}
\newcommand{\tfuture}{\mathcal{F}}
\newcommand{\tuntil}{~\mathcal{U}~}
\newcommand{\tsince}{~\mathcal{S}~}
\newcommand{\tpast}{\mathcal{P}}
\newcommand{\tknows}[1]{\mathcal{K}_{#1}}
\newcommand{\tatmost}[1]{\mathcal{N}_{#1}}
\newcommand{\tpossible}[1]{\mathcal{L}_{#1}}
\newcommand{\tthen}{~\textit{then}~}
\newcommand{\tlast}[2]{\textit{last}(#1, #2)}
\newcommand{\limplies}{\rightarrow}
%\newcommand{\tforall}{\hbox{forall}~}
\newcommand{\texists}{\hbox{exists}~}
\newcommand{\tcurrent}[1]{\hbox{current}_{#1}~}
\newcommand{\tval}[1]{\textit{value}(#1)}
\newcommand{\trelease}{\rhd}
\newcommand{\evt}{\eta}
\newcommand{\mimplies}{\Rightarrow}
\newcommand{\with}{\&}
\newcommand{\tlevel}[3]{\textit{level}(#1, #2, #3)}
\newcommand{\tleveltr}[2]{\textit{level}(#1, #2)}
%\newcommand{\tlevel}[3]{\llbracket #1 \rrbracket_{#2} [#3]}


%% for comments
\newcommand{\comment}[3][\color{red}]{{#1{[{#2}: {#3}]}}}
\newcommand{\kris}[1]{\comment[\color{orange}]{kris}{#1}}
\newcommand{\jeff}[1]{\comment[\color{green}]{JSF}{#1}}
\newcommand{\mrc}[1]{\comment[\color{blue}]{MRC}{#1}}
\newcommand{\jon}[1]{\comment[\color{purple}]{JFD}{#1}}
\newcommand{\jsjeon}[1]{\comment[\color{magenta}]{JJ}{#1}}
\newcommand{\thickhline}{\noalign{\hrule height 1pt}}
\newcommand{\low}{\text{L}\xspace}
\newcommand{\high}{\text{L}\xspace}

\newcommand{\lang}{revea$\lambda$\xspace}
\sloppy
%% XXX: Not sure if I put *my* name or everyone's name
\def\lastname{Please list Your Lastname Here}  

\begin{document}
\begin{frontmatter}
  \title{GUI-Based Declassification Policies \\ for Mobile Apps}
  \author{Kristopher Micinski\thanksref{ALL}\thanksref{micinski@cs.umd.edu}}
  \address{Department of Computer Science\\ University of Maryland,
    College Park\\
    College Park, MD, USA} 

  \author{Jon Fetter-Degges\thanksref{ALL}\thanksref{jonfd@cs.umd.edu}}
  \address{Department of Computer Science\\ University of Maryland,
    College Park\\
    College Park, MD, USA} 

  \author{Jinseong Jeon\thanksref{ALL}\thanksref{jsjeon@cs.umd.edu}}
  \address{Department of Computer Science\\ University of Maryland,
    College Park\\
    College Park, MD, USA} 

  \author{Jeffrey S Foster\thanksref{ALL}\thanksref{jfoster@cs.umd.edu}}
  \address{Department of Computer Science\\ University of Maryland,
    College Park\\
    College Park, MD, USA} 

  \author{Michael Clarkson\thanksref{ALL}\thanksref{clarkson@cs.cornell.edu}}
  \address{Department of Computer Science\\ Cornell University, \\
    Ithaca, NY, USA} 
\begin{abstract} 
  Mobile apps have access to a wide variety of secure information,
  such as contacts and location. However, current mobile platforms
  support only coarse access control mechanisms to protect such
  data. In this paper, we introduce \emph{GUI-based declassification
    policies}, which incorporate LTL to describe precisely what secure
  information is allowed to be released after which series of GUI
  events (e.g., checkbox selects, button clicks, etc). We define our
  policies precisely using a core formalism that models key aspects of
  Android, particularly callbacks and the message queue. Finally, we
  describe a tool that uses symbolic execution to check GUI-based
  declassification policies for Android, and we show that it enforces
  policies correctly on a set of apps.
\end{abstract}
\begin{keyword}
  Security, information flow, Program analysis, symbolic execution.
\end{keyword}
\end{frontmatter}

\section{Introduction}
\label{sec:introduction}

Android apps are incredibly popular, but they pose significant
security concerns by simultaneously having access to both the Internet
and private information such as contacts and location.  Android
includes a permission system, but it only provides access control, and
permissions are only requested at app installation time. To address
these limitations, we wish to develop a much finer-grained security
mechanism that reasons about information flow and that provides tight
control of \emph{when} and \emph{what} sensitive information is
released.

In this paper, we propose \emph{GUI-based declassification policies},
which provide a natural way to specify declassification on Android and
similar mobile platforms. A key hypothesis behind our approach is that
GUI interactions should control information flow
\cite{Roesner:12,Chen:13}. For example, we support policies such as,
``The email address should only be declassified if the send button is
clicked.''  Our policies can also reason about when secrets are
accessed, e.g., ``Only locations read when the \emph{map} option is
selected are declassified.'' To achieve this expressiveness, we define
policies extensionally in terms of app \emph{events}, where an event
is a GUI action (e.g., a button click), a secret input (e.g., a
location update), or an output (e.g., a send over the
Internet). Policies use LTL formulae \cite{Pnueli:1977} to
specify when a particular sequence of events implies that certain
information may be declassified. (Section~\ref{sec:overview} describes
our policies further in the context of Android apps.)

We define our security policies precisely using a core formalism that
models key components of Android. In our formalism, programs are
comprised of handlers that react to events from a \emph{message queue}
(mimicking Android's message queue). The operational semantics is
stratified into a big-step semantics for handler execution and a
small-step semantics that accesses the message queue and generates
\emph{traces} of events. We then interpret a GUI-based
declassification policy as determining the security level of each
event in a trace. Finally, we define noninterference in terms of the
traces at the resulting security levels.  (Section~\ref{sec:formalism}
presents our formalism.)

We developed a prototype symbolic execution tool \toolname{} (Android
GUI Information Flow)
for checking GUI-based declassification policies on
Android apps. Our tool is built using SymDroid~\cite{Jeon:2012}, a Dalvik
bytecode symbolic executor, and Z3~\cite{deMoura:2008}, an SMT solver. We
symbolically execute each app using a model of the Android framework
and using a per-app driver to systematically explore all executions up
to a maximum number of input events. We then pair up the resulting
traces and use Z3 to either discharge the noninterference conditions,
or produce a counterexample showing how the security policy is
violated.  
We ran \toolname{} on several small Android apps, including both
secure and insecure variants, and found it is able to enforce policies
correctly. (Section~\ref{sec:implementation} describes our security
policy checker for Android, and Section~\ref{sec:experiments}
describes our experiments.)

\section{Overview}
\label{sec:overview}

We begin our presentation with two example apps that illustrate
interesting aspects GUI-based declassification policies. We use these
apps as subject programs in Section~\ref{sec:experiments}.

\subsection{Bump app}
The boxed portion of Fig.~\ref{fig:app-bump} gives (simplified)
source code for an Android app that releases a user's email address
and/or phone number. This app is inspired by the Bump app, which
let users tap phones to share selected information with each other.

Each screen of an Android app is implemented using a class that
extends \code{Activity}. When an app is launched, Android invokes the
\code{onCreate} method for one of its activities.
(This is part of the \emph{activity lifecycle}~\cite{AndroidLifecycle}
which includes several methods called in a certain order. For this
simple app, and the other apps used in this paper, we only need a
single activity with this one lifecycle method.)
That method retrieves
(lines~\ref{line:bump-button}--\ref{line:bump-check}) the GUI ids of a
button (marked ``send'') and
two checkboxes (marked ``email'' and ``phone''). The \code{onCreate} method next
gets an instance of the \code{TelephonyManager}, uses it
to retrieve email and phone number information, and unchecks the two
checkboxes. Then it creates a new callback
(line~\ref{line:bump-cb}) to be invoked when the ``send'' button is
clicked. When called, that callback releases the user's email address and/or
phone number depending on the state of the checkboxes.

This app is written to work with \toolname{}, a tool we built to check
whether apps satisfy GUI-based declassification policies. As we
discuss further in Section~\ref{sec:implementation}, our tool uses an
executable model of Android that abstracts away some details that are
unimportant with respect to security. While a real app would release
information by sending it to a web server, here we instead call a
method \code{InfoSender.sendInt}. Additionally, while a real app would
include an XML file specifying the screen layout of buttons,
checkboxes, and so on, \toolname{} simply creates those GUI elements
on demand at calls to \code{findViewById} (since their screen locations are
unimportant). Finally, we model the email address and phone number as
integers to keep the analysis simpler.

\toolname{} simulates paths through subject apps, recording a
\emph{trace} of \emph{events}. Events mark actions that are relevant
for security policies, which in \toolname{} are GUI inputs, secret
inputs, and network sends.  For example, one path through this app
generates a trace
\begin{displaymath}
   \code{email?42}, \code{phone?43}, \code{emailBox?true},
    \code{sendButton?unit}, \code{netout!}\code{42}
\end{displaymath}
Each event begins with a \emph{channel}, which is just an
easy-to-remember name we can refer to in security
policies. \toolname{} includes support for marking GUI elements with
channel names that are then used in events for those elements. For
example, we gave (code not shown) the \code{CheckBox} stored in
\code{emailBox} the homonymous channel name, and similarly for
\code{sendButton}. We also specify channel names for reads from secret
inputs, e.g., \code{email} and \code{phone}. Lastly, we use the
channel name \code{netout} for calls to \code{InfoSender.sendInt}.

In addition to a channel name, each event indicates whether it is
a read (\code{?}) or a write (\code{!}), and carries the value read or
written from the channel. Thus, the trace above indicates
42 is read as the email address, 43 is read as the phone number, the
email checkbox is selected, the send button is clicked (carrying no
particular value, indicated by \code{unit}), and then 42 is
sent on the network.

A GUI-based declassification policy reasons about these traces to
determine what information is allowed to be released. The policy for
this example app is shown at the bottom of Fig.~\ref{fig:app-bump}.
Our policies are comprised of a set of \emph{declassification
  conditions} of the form $\phi \rhd S$, where $\phi$ is an LTL
formula describing event traces and $S$ is a security level.  Such a
condition is read, ``At any input event, if $\phi$ holds at that
position of the event trace, then that input is declassified at level
$S$.''  For this app there are two declassification conditions. The
top condition declassifies (to \emph{Low}) an input that is a
read of the email address at any value (indicated by $\ast$), and if
sometime in the future (indicated by the $\tfuture$ modality) the send
button is clicked and, when that button is clicked, the last value of
the email checkbox was \code{true}. (Note that \emph{last} is not
primitive, but is a macro that can be expanded into regular LTL.)  The
second declassification condition does the analogous thing for the
phone number.

To check such a policy, \toolname{} simulates traces through the program
execution; determines the classification level of every input; and
then checks that every pair of traces satisfies noninterference.

\begin{figure}[t]
\centering
\begin{subfigure}[b]{0.5\textwidth}
  \begin{lstlisting}[name=Ex]
    public class BumpApp extends Activity {
      protected void onCreate(...) {
        Button sendButton = (Button) findViewById(...); /**\label{line:bump-button}*/
        CheckBox emailBox = (CheckBox) findViewById(...);
        CheckBox phoneBox = (CheckBox) findViewById(...);/**\label{line:bump-check}*/
        TelephonyManager manager = 
        TelephonyManager.getTelephonyManager(); /**\label{line:bump-tele}*/
        final int email = manager.getEmail(); /**\label{line:bump-email}*/
        final int number = manager.getPhoneNumber();
        emailBox.setChecked(false);
        phoneBox.setChecked(false);
        sendButton.setOnClickListener(new OnClickListener() { /**\label{line:bump-cb}*/
          public void onClick(View v) {
            if (emailBox.isChecked())
            InfoSender.sendInt(email); 
            if (phoneBox.isChecked())
            InfoSender.sendInt(number);
          } });
      } }
  \end{lstlisting}
\end{subfigure}
\begin{subfigure}[b]{0.5\textwidth}
  \begin{displaymath}
    \begin{array}{c}
      \code{email}?\ast \wedge \big(\tfuture ( \code{sendButton?unit} \land\;
      \tlast{\code{emailBox}}{\code{true}})\big) \rhd Low, \\

      \code{phone}?\ast \wedge \big(\tfuture ( \code{sendButton?unit} \land\;
      \tlast{\code{phoneBox}}{\code{true}})\big) \rhd Low \\
    \end{array}
  \end{displaymath}
\end{subfigure}
\caption{``Bump'' app, which shares either email or phone number.}
\label{fig:app-bump}
\end{figure}
 
Notice that using LTL in our policies gives us a very general and
expressive way to describe the sequences of events that imply
declassification. For example, here we precisely capture that
only the last value of the checkbox matters for declassification. For
example, if a user selects the email checkbox but then unselects it
and clicks send, the email is not allowed to be released.

Also notice that this policy depends on the app reading the email
address and phone number when the app starts. If the app instead
waited until after the send button is clicked, it would violate this
policy. We could address this by replacing the $\tfuture$ modality by
$\tpast$ (past) in the policy (and we could form a disjunction of the
two policies if we wanted to allow either).

The reason our policy is sensitive to this implementation choice is
that our framework is designed to support reasoning about secret
values that change over time, so we can precise specify the level of a
secret input that was read at a particular time. We will see an
example of this next.

% \begin{figure}[t!]
% \begin{lstlisting}[name=Ex]
% public class ShareContact extends Activity {
%   private Contact mCurrentContact;
%   protected void onCreate(Bundle savedInstanceState) {
%     Spinner contactSpinner = (Spinner) findViewById(...);
%     SpinnerAdapter adapter = 
%       ContactAdapterFactory.makeAdapterFactory(this);
%     contactSpinner.setAdapter(adapter);
%     contactSpinner.setOnItemSelectedListener(
%       new OnItemSelectedListener() {
%         public void onItemSelected(AdapterView<?> par, 
%           View v, int pos, long id) {
%         mCurrentContact = 
%           parent.getItemAtPosition(position);
%         }
%         public void onNothingSelected(AdapterView<?> p) {
%           mCurrentContact = null;}
%       });
%     Button shareButton = (Button) findViewById(...);
%     shareButton.setOnClickListener(new OnClickListener() {
%       public void onClick(View v) {
%         if (currentContact != null) 
%           InfoSender.sendInt(mCurrentContact.getNumber());
%     } });
% } }
% \end{lstlisting}

% \begin{displaymath}
%   \begin{array}{c}
%     \code{con_1}? \land \tfuture \big( \code{send}=() \land
%     last(\code{s})=1 \big) \rhd L \\
%     \code{con_2}? \land \tfuture \big( \code{send}=() \land
%     last(\code{s})=2 \big) \rhd L \\ 
%     \code{con_3}? \land \tfuture \big( \code{send}=() \land
%     last(\code{s})=3 \big) \rhd L \\ 
%   \end{array}
% \end{displaymath}

% \caption{Contact picker app.}
% \label{fig:app-contact}
% \end{figure}

\begin{figure}[t!]
\begin{lstlisting}[name=Ex]
public class ToggleResolution extends Activity { ...
  LocSharer mLocSharer = new LocSharer();
  RadioManager mRadio = new RadioManager();
  protected void onCreate(...) {
    ...
  }
  private class LocSharer implements
    LocationListener { ... 
    public LocSharer(RadioManager rm) {
      lm = (LocationManager) 
        getSystemService(Context.LOCATION_SERVICE);
      lm.requestLocationUpdates(mCurrentProvider, 
        SHARE_INTERVAL, distance, this);
    }
    public void onLocationChanged(Location l) {
      if (mRadio.mFine) {
        InfoSender.sendInt(l.mLatitude);
        InfoSender.sendInt(l.mLongitude);
      } else {
        InfoSender.sendInt(l.mLatitude & 0xffffff00);
        InfoSender.sendInt(l.mLongitude & 0xffffff00);
  } } }
  private class RadioManager implements OnClickListener {
    public boolean mFine = false;
    public void onClick(View v) {
      mFine = !mFine;
} } }
\end{lstlisting}

\begin{displaymath}
  \begin{array}{c}
    \code{longitude}?\ast \;\wedge\; \tlast{\code{mRadio}}{\code{"fine"}} \rhd
    \textit{Low}, \\
    \code{longitude}?\ast \;\wedge\; \tlast{\code{mRadio}}{\code{"coarse"}} \rhd
    \textit{MaskLower8}
  \end{array}
\end{displaymath}
\caption{Location sharing app with varying resolution.}
\label{fig:app-loc-toggle}
\end{figure}

\subsection{Location resolution toggle app}

Fig.~\ref{fig:app-loc-toggle} gives source code for an app that
shares location information, either at full or truncated resolution
depending on the setting of a radio button. The app's \code{onCreate}
method displays a radio button (code not shown) and then creates and
registers a new instance of class \code{RadioManager} to be called
each time the radio button is changed. That
class maintains a field \code{mFine} as \code{true} when the radio button is
set to full resolution and false when it is set to truncated
resolution.

Separately, \code{onCreate} registers \code{LocSharer} to be called
periodically with the current location.  It requests location updates
by registering itself as a callback with the \code{LocationManager}
system service.  Whenever it is called, \code{LocSharer} releases the
current location, either at full resolution or with the lower 8 bits
masked off, depending on the setting of \code{mFine}.

The declassification policy for longitude appears below the code; the
policy for latitude is analogous.  This policy allows the precise
value of the longitude to be released when the last selection for
\code{mRadio} was the fine setting, but only the lower eight bits to
be released if \code{mRadio} was last set to coarse. Here \toolname{}
knows that at the \textit{MaskLower8} level, it should consider
outputs to be equivalent up to differences in the lower 8
bits. 

Finally, notice that this policy does not use the future
modality. This is deliberate, because location may be read multiple
times during the execution, at multiple values. Thus, we want to assign different security
levels to different reads of location, depending on the state of the
radio button when they were read. For example, consider a trace
\begin{displaymath}
   \code{mRadio?false}, \code{longitude?}v_1,
    \code{mRadio?true},\code{longitude?}v_2\\
\end{displaymath}

The second declassification condition ($\code{longitude}?\ast \wedge
\tlast{\code{mRadio}}{\code{false}}$) will match for $v_1$, since
the last value on the \code{mRadio} channel was \code{false}, and
thus $v_1$ may be declassified only to \textit{MaskLower8}. Whereas
the first declassification condition will match for $v_2$, hence it
may be declassified to \textit{Low}.

\section{Formalizing Android}
\label{sec:formalism}

The examples in Section~\ref{sec:overview} highlighted several
language features that are relevant to GUI-based declassification
policies. Apps register callbacks that are invoked when a GUI element
is clicked.  Secret information comes in either via direct calls
(e.g., \code{manager.getEmail} in Fig.~\ref{fig:app-bump}) or
callbacks (e.g., \code{onLocationChanged} in
Fig.~\ref{fig:app-loc-toggle}).  Callbacks are orchestrated by a
\emph{message queue}, which globally sequences these events.
% The message queue is also used when one
%\code{Activity} invokes another, which happens whenever an app
%switches from one screen to another.  To generate new events (which
%will be handled by registered callbacks), methods generate messages to
%be placed on this queue (e.g., by sending an \code{Intent} object in
%Android: a message that allows a set of actions to be taken).

In this section, we present a core Android formalism that captures the
features just mentioned and abstracts away other details that are less
important in our setting.  Programs in our language comprise handlers
that receive messages from the queue and optionally send new
messages. Programs are executed by repeatedly pulling messages off the
queue and running the corresponding handler. User and secret inputs
and injected non-deterministically into the queue. The whole process
generates a trace of events, which we use to define GUI-based
declassification policies.

% Then we define GUI-based declassification policies for our programs
% using a formal grammar, and define program security in terms of the
% output of the semantics.


\subsection{Formal language}

\begin{figure}[t!]
  \centering
  \begin{subfigure}{0.5\textwidth}
    \begin{displaymath}
      \begin{array}{lrcl}
        \hbox{prims} & \xv & ::= & n \mid \strue \mid \sfalse \mid f(\xv_1, \ldots, \xv_i) \\
        \hbox{values} & v & ::= & p \mid \loc \mid \lambda x.e \\
        \hbox{exprs} & e & ::= &
        v
        \mid x
        \mid e_1~e_2
        \mid  \sref{e}
        \mid \sassign{e_1}{e_2}
        \mid \;\sderef{e} \\
        && \mid & \sif{e_1}{e_2}{e_3}
        \mid e_1 \oplus e_2 \\
        && \mid & f(e_1, \ldots, e_n)
        \mid f^{-n}(e) \\
        && \mid & \sinstall{\sch}{e}
        \mid \ssend{\sch}{e} \\
        \hbox{ops} & \oplus & ::= &  = \mid < \mid > \mid \wedge \mid \ldots \\
      \end{array}
    \end{displaymath}
  \end{subfigure}
  \begin{subfigure}{0.5\textwidth}
    \begin{displaymath}
      \begin{array}{rclp{1in}}
        \Sigma & = & (M, \sigma, H) & State \\
        M & : & (\sch \times p) ~ \textit{list} & Message queue \\
        \sigma & : & \loc \partialfun  v & Heap \\
        H & : & \sch \partialfun \lambda x.e & Handler map \\
        \evt & : & \sch ? p \mid \sch ! p \mid \tau & Event \\
        t & : & \evt ~ \textit{list} & Trace \\
        % S & : & \sch \partialfun n \rightarrow p & Secrets
      \end{array}
    \end{displaymath}
  \end{subfigure}
  \caption{Source language syntax and semantic definitions.}
  \label{fig:lang}
\end{figure}

Fig.~\ref{fig:lang} presents our source language.
\emph{Primitives}~$p$ are terms that can be sent over a channel:
integers, booleans, and terms constructed from primitives using
uninterpreted constructors $f$.  \emph{Values}~$v$ are fully reduced terms in
the semantics, and may be either primitives, \emph{locations}~$\loc$
(which never appear in the surface syntax), or functions.

\emph{Expressions}~$e$ include values~$v$, variables~$x$,
function applications~$e_1~e_2$, memory
allocation~$\sref{e}$, assignment~$\sassign{e_1}{e_2}$, and
dereference~$\sderef{e}$. Expressions also include a conditional form
$\sif{e_1}{e_2}{e_3}$, binary operations $e_1\oplus e_2$, constructed
terms $f(e_1, \ldots, e_n)$, and projection $f^{-n}(e)$ which selects the
$n$th element of a term constructed from $f$.

Expression $\sinstall{\sch}{e}$ installs $e$ (which must evaluate to a
lambda) to receive messages on a \emph{channel} $\sch$. Channels are
used for callbacks, network sends, and to carry secret inputs. We assume
all channels have global scope, and channels are not first class
(they cannot be results of expression evaluation).

\paragraph*{Example insecure application}

\begin{figure}[t!]
  \lstset{language=Caml}
  \begin{lstlisting}[name=Ex]
(* Channels:
   emailBox - email checkbox
   phoneBox - phone checkbox
   sendButton - send button
   email - email secret input
   phone - phone secret input
   netout - network output *)

let email_release = ref false in
let phone_release = ref false in
let email = ref "" in
let phone = ref "" in
let handle_b () = /**\label{line:handle-btn}*/
  if !email_release send netout !phone else unit /**\label{line:mal-email}*/
  if !phone_release send netout !email else unit /**\label{line:mal-phone}*/
in
let handle_c r v = r := v in
  (* onCreate *)
  install email (fun x -> email := x) /**\label{line:store-email}*/
  install phone (fun x -> phone := x) /**\label{line:store-phone}*/
  install emailBox (handle_c email_release);
  install phoneBox (handle_c phone_release);
  install sendButton handle_b
\end{lstlisting}
  \caption{An insecure variant of the bump app.}
  \label{fig:evilappexample}
\end{figure}

Fig.~\ref{fig:evilappexample} shows a translation of the app in
Fig.~\ref{fig:app-bump} into our formal language, but with a malicious
twist. In this translation, we use standard OCaml \code{let} syntax,
which can be desugared into lambdas.  The app assumes the existence of
several channels, listed at the top of the figure, for GUI input,
secret input, and network output. The app is written as an expression
that is run when the application is launched. It begins by
installing handlers to retrieve the email and phone number secrets and
store them in references (lines~\ref{line:store-email}
and~\ref{line:store-phone}). It then installs handlers that set flags
\code{email_release} and \code{phone_release} according to the
corresponding checkbox state (note here we use partial application of
\code{handle_c} to accomplish this). Lastly, it installers
\code{handle_b} as a callback for the send button. When the send
button is clicked, \code{handle_b} checks to see whether the email
address has been declassified and, if so, it releases the phone number
(line~\ref{line:mal-email}) and vice versa
(line~\ref{line:mal-phone}), resulting in the malicious behavior.

\subsection{Operational Semantics}

The bottom portion of Fig.~\ref{fig:lang} presents some other notation used
in our semantics. A \emph{state} is a tuple
$(M, \sigma, H)$ that includes a \emph{message
  queue}~$M$, which is a list of pairs of a channel and a primitive; a
\emph{heap}~$\sigma$, which maps locations to values; and a \emph{handler
  map}~$H$, which maps channel names to functions installed to
handle events on those channels. Each channel may have at most one handler.
We write $\Sigma.X$ (where $X$ could be $M$,
$\sigma$, or $H$), to mean the $X$ component of
$\Sigma$. Similarly, we write $\Sigma[X\mapsto X']$ to mean $\Sigma$
with the $X$ component replaced by $X'$.

% I dropped ``externally observable'' because the secret inputs are
% *not* externally observable.
As programs execute, they produce a \emph{trace}~$\tr$ of 
\emph{events}~$\evt$. Events are reads and writes,
written $\sch ? p$ and $\sch ! p$, respectively, of primitive $p$
from channel $\sch$. We also include an empty event
$\tau$, which is the identity of trace concatenation.

\begin{figure*}[!t]
  \small
  \begin{displaymath}
    \begin{array}{c}
      \multicolumn{1}{l}{
        \framebox{$e, \Sigma_1 \sreduce v, \Sigma_2$}
%      \qquad
%      \hbox{\textit{In state $\Sigma_1$, expression $e$ reduces to
%      value $v$ and new state $\Sigma_2$.}}
      }
      \\ \\

      \infer[RVal]
      { }
      { v, \Sigma \sreduce v, \Sigma }

      \qquad

      \infer[RApp]
      {
        e_1, \Sigma_1 \sreduce (\lambda x.e_3), \Sigma_2 \\
        e_2, \Sigma_2 \sreduce v_1, \Sigma_3 \\\\
        e_3\aset{x\mapsto v_1}, \Sigma_3 \sreduce v_2, \Sigma_4
      }
      { e_1\;e_2, \Sigma_1 \sreduce v_2, \Sigma_4}

      \qquad

      \infer[RRef]
      {e, \Sigma_1 \sreduce v, \Sigma_2 \\
        \loc \not\in \dom(\Sigma_2.\sigma) \\\\
        \sigma' = (\Sigma_2.\sigma)[\loc\mapsto v] 
      }
      {\sref e, \Sigma \sreduce \loc, \Sigma_2[\sigma \mapsto \sigma']}

      \\ \\

      \infer[RAssign]
      {e_1, \Sigma_1 \sreduce \loc, \Sigma_2 \\
        \loc \in \dom(\Sigma_2.\sigma) \\\\
        e_2, \Sigma_2 \sreduce v, \Sigma_3 \\
        \sigma' = (\Sigma_3.\sigma)[\loc \mapsto v]
      }
      {\sassign {e_1} {e_2}, \Sigma_1 \sreduce
        v, \Sigma_3[\sigma \mapsto \sigma']}


      \qquad

      \infer[RDeref]
      {e, \Sigma_1 \sreduce \loc, \Sigma_2 \\
       (\Sigma_2.\sigma)(\loc) = v }
      {\sderef e, \Sigma_1 \sreduce v, \Sigma_2 }

      \\ \\

      \infer[RIfTrue]
      {e_1, \Sigma_1 \sreduce \code{true}, \Sigma_2 \\
      e_2 , \Sigma_2 \sreduce v , \Sigma_3}
      {\sif{e_1}{e_2}{e_3}, \Sigma_1 \sreduce v, \Sigma_3}
      
      \qquad

      \infer[RIfFalse]
      {e_1, \Sigma_1 \sreduce \code{false}, \Sigma_2 \\
        e_3 , \Sigma_2 \sreduce v , \Sigma_3}
      {\sif{e_1}{e_2}{e_3}, \Sigma_1 \sreduce v, \Sigma_3}
      
      \\ \\

      \infer[ROp]
      {e_1, \Sigma_1 \sreduce v_1, \Sigma_2 \\
       e_2, \Sigma_2 \sreduce v_2, \Sigma_3}
      {e_1 \oplus e_2, \Sigma_1 \sreduce v_1 \oplus v_2, \Sigma_3}

      \qquad

      \infer[RCstr]
      {e_i, \Sigma_i \sreduce v_i, \Sigma_{i+1} \\ i\in 1..n}
      {f(e_1, \ldots, e_n), \Sigma_1 \sreduce f(v_1, \ldots, v_n), \Sigma_{i+1}}

      \qquad

      \infer[RProj]
      {e, \Sigma_1 \sreduce f(v_1, \ldots, v_n), \Sigma_2 }
      {f^{-i}(e), \Sigma_1 \sreduce v_i, \Sigma_2}

      \\ \\

      \infer[RInst]
      {
        e_1, \Sigma_1 \sreduce (\lambda x.e_2), \Sigma_2 \\
        H' = (\Sigma_2.H)[\sch \mapsto \lambda x.e_2]
      }
      {
        \sinstall \sch {e_1}, \Sigma_1 \sreduce \sunit, \Sigma_2[H
        \mapsto H']
      }

      \qquad

      \infer[RSend]
      { e, \Sigma_1 \sreduce \xv, \Sigma_2 \\
        M' = (\Sigma_2.M) @ (\sch, \xv)
      }
      { \ssend \sch e, \Sigma_1 \sreduce \sunit, \Sigma_2[M \mapsto M']}
      \\ \\

      \\ \\ 

      \multicolumn{1}{l}{
        \framebox{$\Sigma_1 \treduce^{\evt} \Sigma_2$ \hbox{ and } 
          $\judge e \rightsquigarrow t$}
      }
      \\ \\

      \infer[THandle]
      { \Sigma_1.M = (\sch, \xv) @ M' \\
        \Sigma_1.H(\sch) = \lambda x.e \\\\
        \Sigma' = \Sigma_1[M\mapsto M'] \\
        e \aset{x \mapsto \xv} , \Sigma' \sreduce v, \Sigma_2 \\
      }
      { \Sigma_1 \treduce^{\tau} \Sigma_2}

      \qquad

      \infer[TInput]
      { \Sigma' = \Sigma[M \mapsto (\Sigma.M) @ (\sch, p)] }
      { \Sigma \treduce^{\sch?p} \Sigma' }

      \\ \\ 

      %% \infer[TInput]
      %% { \Sigma' = \Sigma[M \mapsto (\Sigma.M) @ (\sch , p)] \\\\
      %%   \sch \in \dom(\Sigma.H)
      %% }
      %% { \Sigma \treduce^{\sch?p@(\Sigma.\Phi)} \tick{\Sigma'} }

      \infer[TOutput]
      { \Sigma.M = (\sfmt{netout},p), M' \\\\
        \Sigma' = \Sigma[M\mapsto M']
      }
      { \Sigma \treduce^{\sfmt{netout}!p} \Sigma'}

      \qquad

      % \infer[TLookupSecret]
      % { \Sigma.M = (\sfmt{getsecret},\sch^s,i), M' \\\\
      %   \Sigma' = \Sigma[M\mapsto M' @ (\sch, \Sigma.S(\sch)(\Sigma.i))]
      % }
      % { \Sigma \treduce^{(\sch, \Sigma.S(\sch)(\Sigma.i))} \Sigma'}
      
      % \\ \\ 

      \infer[TProg]
      {
      \Sigma_0 = ([(\code{onCreate},\code{unit})]
                  ,\emptyset
                  ,\aset{\code{onCreate} \mapsto
                          \lambda x . e})
      \quad x \not\in FV(e) 
        \\\\ 
      \Sigma_i \treduce^{\evt_i} \Sigma_{i+1}
      \quad i \in [0..n]
      }
      { \judge e \rightsquigarrow \evt_0 \cdot \evt_1 \cdots
      \evt_n }
    \end{array}
  \end{displaymath}
  \caption{Semantics for our Android subset.}
  \label{fig:semantics}
\end{figure*}

Our semantics is stratified into two levels: a big-step semantics,
shown at the top of Fig.~\ref{fig:semantics}, which models evaluation of code in a
handler, and a small-step semantics, shown at the bottom of the
figure, which models the message queue.  The semantics generates sets
of traces containing input and output messages.

% \mrc{heap update notation isn't defined}
% I removed this comment because we noted in the previous paragraph
% that we use the notation \Sigma[X|->X'] to mean \Sigma with X
% component replaced by X' and we use that for heap updates as they're
% part of the state.

The big-step semantics proves judgments of the form
$e, \Sigma_1 \sreduce v, \Sigma_2$, meaning
evaluation of expression $e$ in state $\Sigma_1$ produces a value $v$ and
new state $\Sigma_2$. 
The first several rules are standard.
\textsc{RVal} evaluates a value to itself, without changing the
state. \textsc{RApp} evaluates $e_1$ to a lambda, evaluates $e_2$ to
a value, and then evaluates the body of the lambda with the actual
argument substituted for the formal variable: we use the notation
$e\aset{x\mapsto v}$ for $e$ when unbound occurrences of $x$ in $e$
have been syntactically replaced with $v$. 
\textsc{RRef} evaluates $e$ to a value $v$, finds a fresh location
$\loc$ in the heap, and then evaluates to $\loc$, returning a state
where the heap maps $\loc$ to $v$. \textsc{RAssign} evaluates $e_1$
to a location $\loc$ and then updates the contents of
$\loc$. \textsc{RDeref} evaluates $e$ to a location and returns the
contents of that location.

\textsc{RIfTrue} evaluates $e_1$ to a value and if it evaluates to
\code{true}, evaluates $e_2$. \textsc{RIfFalse} is
analogous.
\textsc{ROp} evaluates a binary operation by applying the designated
operation to the values of the two subexpressions.
%JF: I don't think we need to define this precisely.
% using the denotation
% function for $\oplus$ which we assume is provided. \kris{TODO: Define
%   denotation / interpretation for $\oplus$.}

\textsc{RCstr} and \textsc{RProj} construct terms and project from
constructed terms. Finally, \textsc{RInst} evaluates $e_1$ to a lambda
and adds it as a handler for channel $\sch$; the result value is the
unit value $\sunit$ (a nullary constructor). \textsc{RSend} evaluates
$e$ to a primitive $\xv$, and then adds the message $(\sch,\xv)$ to the
end of the message queue. Here we use $@$ for concatenation. Note that \textsc{RSend} only allows
primitives to be sent, and not locations or lambda
expressions.

The first three rules 
in small-step semantics at the bottom of
Fig.~\ref{fig:semantics} prove judgments of the form
$\Sigma_1 \treduce^{\evt} \Sigma_2$, meaning the machine can take a step from state
$\Sigma_1$ to a new state $\Sigma_2$, producing an event $\evt$.
\textsc{THandle} consumes a message $(\sch, \xv)$ from the front of
the message queue (recall \textsc{RSend} adds a message to the
\emph{end}  of the message queue), looks up the handler for $\sch$,
and then invokes the handler, passing $\xv$ as its argument. Running a
handler is not an externally visible operation, which is indicated by an empty
event $\tau$ on the reduction arrow. 

\textsc{TInput} models an input message, which may be due to user
input (e.g., GUI clicks) or secret input from the system (e.g.,
a callback with updated location information). This rule
non-deterministically picks some channel $\sch$ and an arbitrary
primitive $p$, and then sends $p$ on that channel. 
The input is recorded as an event on the reduction arrow.
Notice that here we do not distinguish the security level of an input---we choose to leave that
up to the security policy designer, who can opt to either always
designate GUI inputs as low-security (as we do in our experiments) or
make them high-security.

\textsc{TOutput} models writes to the network, consuming a message
from a distinguished channel \sfmt{netout}. Writing to this channel
corresponds to the \code{InfoSender.sendInt} calls in
Section~\ref{sec:overview}. Since these messages may be seen by the
observer---i.e., they are ``low visible'' outputs---we record the
write event on the reduction arrow. Note that by convention
there is no user handler for this channel.

Finally, \textsc{TProg} proves a judgment of the form
$\judge e \rightsquigarrow t$, meaning running the program $e$ produces a
trace $t$ of events.
This rule creates an initial state $\Sigma_0$ in
which $e$ is bound as a handler on a special \code{onCreate} channel,
and the message queue contains an initial message on that channel. The
rule then repeatedly steps to the next state $n+1$ times. It produces 
the event trace $\eta_0\cdots\eta_n$.
Notice that the length of the trace $n$ is nondeterministic; in general,
since these are reactive programs, they can usually run for an any
number of steps as long as additional input arrives.

From the set of all executions of a program we can extract a set of
traces, which we later use to define noninterference.
\begin{definition}[Program Traces]
  We define the set of traces of a program $e$ as
\begin{displaymath}
   \traces(e) = \aset{t \mid \; \judge e { \rightsquigarrow t }}
\end{displaymath}
\end{definition}

For example, consider the app in Fig.~\ref{fig:evilappexample}. Among
others, it generates the following two traces:

\begin{displaymath}
  \begin{array}{cl}
    \code{email}?0, \code{phone}?0, \code{emailBox}?\code{true},
    \code{sendButton}?\code{unit}, \code{netout}!0 & (1) \\
    \code{email}?0, \code{phone}?1, \code{emailBox}?\code{true},
    \code{sendButton}?\code{unit}, \code{netout}!1 & (2) \\
  \end{array}
\end{displaymath}
\lstset{language=Java}
In the first trace, the email address and phone number are read as 0, the
email checkbox is selected, the button is clicked, and 0 is sent. In
the second trace is similar, except the phone number is 1, and 1 is
sent. Below, we will use these trace to show that this program
violates its security policy.

\subsection{GUI-based Declassification Policies}
\label{sec:policies}

% Our example policies (given in section \ref{sec:examples}) used a
% reduced syntax for specifying declassification policies.  We now make
% this syntax formal and demonstrate how it gives extensional gaurantees
% about information flow properties of programs.  We state information
% policies for programs in the logic \hyperltltwo \cite{Clarkson:2014}.
% \hyperltltwo allows quantification over paths, and atomic predicates
% that speak about the contents of paths.


We can now define our policy language precisely.
Fig.~\ref{fig:policy-temporal-logic} gives the formal syntax of
declassification policies.  A policy $P$ is a set of
\emph{declassification conditions} $C_i$ of the form $\phi_i\rhd S_i$,
where $\phi_i$ is an LTL formula describing when an input is
declassified, and $S_i$ is a \emph{security level} at which the value
read at that input event is declassified.

As is standard, security levels $S$ form a lattice. 
For our framework, we require that this lattice be finite.
We include
\textit{High} and \textit{Low} security levels, and we can generalize to
arbitrary lattices in a straightforward way. Here we include the
\textit{MaskLower8} level from Fig.~\ref{fig:app-loc-toggle} as an
example, where $\textit{Low} \sqsubseteq \textit{MaskLower8} \sqsubseteq \textit{High}$.
Note that although we include \textit{High} in the language, in practice there
is no reason to declassify something to level \textit{High}, since then it
remains secret.

The \emph{atomic predicates}~$A$ of our LTL formulae match trace read
or write events, e.g., atomic predicate $\sch?p$ is true of an event that is a
read of $p$ on channel $\sch$. We include $\ast$ for matches to
arbitrary primitives. We allow the read or written values to be
variables that are bound in an enclosing quantifier. The atomic
predicates also include the ability to make atomic arithmetic
statements.  We assume that the interpretation of $\oplus$ is the same
as in the semantics of figure~\ref{fig:lang}.
The combination of these lets us describe complex input
events. For example, we could write
$\exists x. \textit{spinner}?x \wedge x > 2$ to indicate an input
event that read the \emph{spinner} channel with a value greater than
2.

Atomic predicates can be combined with the usual boolean connectives
($\neg$, $\wedge$, $\vee$, $\rightarrow$) and existential and
universal quantification.  Our formulae also include the standard LTL
modalities $\tnext$ (next), $\tuntil$
(until), $\talways$ (always), $\tfuture$ (future), $\phi \tsince \psi$
(since), and $\tpast \phi$ (past).  We also include
$\tlast{\code{ch}}{p}$, which is syntactic sugar for $\lnot (ch?\ast)
\tsince \code{ch}?p$.

\begin{figure}[!t]
  \small
  \begin{displaymath}
    \begin{array}{rcl}
      P & ::= & C_1, C_2, \ldots \\
      C & ::= & \phi \rhd S\\
      S & ::= & \textit{High} \mid \textit{Low} \mid
      \textit{MaskLower8} \mid \ldots \\ % \mid Parity \mid ... \\
      % \psi & ::= &       \atom
      % \mid \neg \psi
      % \mid \psi \wedge \psi
      % \mid \psi \vee \psi
      % \mid \psi \limplies \psi
      % \mid \exists x.\psi 
      % \mid \forall x.\psi 
      %              \\
      \atom & ::= & \sch?s \mid \sch!s \mid s \oplus s \\
      s & ::= & x \mid p \mid \ast \\
      \phi & ::= &
      \atom
      \mid \neg \phi
      \mid \phi \wedge \phi
      \mid \phi \vee \phi
      \mid \phi \limplies \phi
      \mid \exists x.\phi 
      \mid \forall x.\phi \\
      & \mid &  \tnext \phi
      \mid \phi \tuntil \phi
      \mid \talways \phi 
      \mid \tfuture \phi 
      \mid \phi \tsince \phi
      \mid \tpast \phi
                         \\
    \end{array}
  \end{displaymath}

  %% \begin{displaymath}
  %%   \begin{array}{rcl}
  %%     \tr , i \models \sch\{!,?\}p_1 & \iff & \tr[i] = \sch\{!,?\}p_1 \\
  %%     \tr , i \models R(p_1, \ldots , p_n) & \iff & R(p_1, \ldots , p_n) \text{~is
  %%     valid in arithmetic} \\
      
  %%     \tr, i \models \neg \phi & \iff &
  %%     \tr, i  \not \models \phi \\

  %%     \tr, i  \models \phi \land \psi & \iff &
  %%     \tr, i  \models \phi \hbox{ and } \tr, i  \models \psi \\

  %%     \tr, i  \models \phi \lor \psi & \iff &
  %%     \tr, i  \models \phi \hbox{ or } \tr, i  \models \psi \\

  %%     \tr, i  \models \phi \limplies \psi & \iff &
  %%     \tr, i  \models \psi \hbox{ or } \tr, i  \models
  %%     \lnot \phi \\

  %%     \tr, i  \models \talways \phi & \iff &
  %%     \hbox{for all } j ~.~ j \geq i \mimplies (\tr, j  \models \phi) \\

  %%     \tr, i  \models \tfuture \phi & \iff &
  %%     \hbox{there exists } j, j \geq i \mimplies (\tr, j  \models \phi) \\

  %%     \tr, i  \models \tpast \phi & \iff &
  %%     \hbox{there exists } j, j \leq i \mimplies (\tr, j  \models \phi) \\

  %%     \tr, i  \models \phi \tuntil \psi & \iff &
  %%     \hbox{there exists } j, j \geq i \mimplies ((\tr, j  \models \psi)
  %%     \hbox{ and } \\
  %%     & & \hbox{for all } k, i \leq k < j \mimplies (\tr, k 
  %%     \models \phi)) \\

  %%     \tr, i  \models \phi \tsince \psi & \iff &
  %%     \hbox{there exists } j, j \leq i \mimplies ((\tr, j  \models \psi)
  %%     \hbox{ and } \\
  %%     & & \hbox{for all } k, j < k \leq i \mimplies (\tr, k 
  %%     \models \phi)) \\

  %%     \tr, i  \models \forall x. \phi & \iff &
  %%     \hbox{for all } p, (\tr, i  \models \aset{x\mapsto p} \phi) \\

  %%     \tr, i  \models \exists x. \phi & \iff &
  %%     \hbox{there exists } p, (\tr, i  \models \aset{x\mapsto p} \phi) \\
  %%   \end{array}
  %% \end{displaymath}

  % \begin{displaymath}
  %   \begin{array}{c}
  %     L \sqsubseteq H \qquad \bigsqcap \{ \} = H
  %   \end{array}
  % \end{displaymath}
  
  % \begin{displaymath}
  %   \begin{array}{c}
  %     \llbracket \tr \rrbracket_P [i] = 
  %     \bigsqcap 
  %     \Big \{ S_j \Big| 
  %     \tr , i \models \phi_j
  %     \Big \}
  %   \end{array}
  % \end{displaymath}

  \caption{GUI-based Declassification Policy Language.}
  \label{fig:policy-temporal-logic}
\end{figure}
% As an example, the policy for the bump program is: 

% \begin{displaymath}
%   \begin{array}{c}
%     \code{email}?v \land \big( \tfuture ( \code{button} = () \land 
%     last(\code{email\_released})=\code{true} \big) \rhd L, \\

%     \code{email}?v \land \big( \tfuture ( \code{button} = () \land 
%     last(\code{email\_released})=\code{true} \big) \rhd L, \\
%   \end{array}
% \end{displaymath}

% Policies associate inputs with security levels, which roughly assert
% what an observer is allowed to learn about that input.  Our system has
% a finite (but arbitrarily large) set of security levels (with
% metavariable $L$ in figure~\ref{fig:policy-temporal-logic}) with at
% least the levels $H$ (high security) and $L$ (low security).  Low
% security inputs are public: the observer is allowed to know their
% values.  High security inputs are private: the observer is never
% allowed to know anything about their values.  This is subsequently
% made formal in section~\ref{}.

We assume a standard interpretation of LTL formulae over
traces \cite{Lichtenstein:85}.
We write $\tr, i \models \phi$ if trace $\tr$ is a model of $\phi$ at
position $i$ in the trace.

Next consider a trace $\tr$ generated by the semantics of
Fig.~\ref{fig:semantics}. We write $\tlevel{\tr}{P}{i}$ for the
security level that policy $P$ assigns to the event $\tr[i]$:

\begin{displaymath}
  \tlevel{\tr}{P}{i} =
  \begin{cases}
    \bigsqcap_{\phi_j\rhd S_j \in P} \aset{ S_j \mid \tr, i \models
      \phi_j } & \tr[i] = \sch?p \\
    \textit{Low} & \tr[i] = \sch!\code{netout} \\
  \end{cases}
\end{displaymath}

In other words, for inputs, we take the greatest lower bound (the most
declassified) of the levels from all declassification conditions that
apply. We always consider outputs on the network output channel to be
declassified. Notice that if no policy applies, the level is $H$ by
definition of greatest lower bound.

For example, consider trace (1) above from the app in
Fig.~\ref{fig:evilappexample} with respect to the policy in
Fig.~\ref{fig:app-bump}.  At position 0, the LTL formula holds because
the email box is eventually checked and then the send button is
clicked, so $\tlevel{(1)}{P}{0} = \textit{Low}$. On the other hand,
$\tlevel{(1)}{P}{1} = \textit{High}$ because no declassification
condition applies for the \code{phone} input (\code{phoneBox} is never
checked). And $\tlevel{(1)}{P}{4} = \textit{Low}$, because that
position is a network send.

Next consider applying this definition on the GUI inputs. As written,
we have $\tlevel{(1)}{P}{2}$ = $\tlevel{(1)}{P}{3}$ =
\textit{High}. However, our app is designed to leak these inputs. 
For example, an adversary will clearly learn the state of
\code{emailBox} if they receive a message with an email address. Thus,
for all the subject apps in this paper, we also declassify all GUI inputs as
\textit{Low}. 
For the example in Fig.~\ref{fig:app-bump}, this means
adding the conditions
$\code{emailBox?}\ast \rhd \textit{Low}$,
$\code{phoneBox?}\ast \rhd \textit{Low}$, and
$\code{sendButton?}\ast \rhd \textit{Low}$. In general, it is up to
the security policy designer to decide the security level of GUI inputs.

Next, we can apply \textit{level} pointwise across a trace and discard
any trace elements that are below a given level $S$. We define
\begin{displaymath}
\tleveltr{\tr}{P}^S[i] =
\begin{cases}
\tr[i] & \tlevel{\tr}{P}{i} \sqsubseteq S \\
\tau & \textrm{otherwise}
\end{cases}
\end{displaymath}
We write $\tleveltr{\tr}{P}^{S,in}$ for the same filtering, except
always discarding output events.

Considering the traces (1) and (2) again, we have
\begin{displaymath}
  \begin{array}{r@{ }c@{ }l}
    \tleveltr{(1)}{P}^\textit{Low} & = & \code{email}?0, \code{emailBox}?\code{true},
    \code{sendButton}?\code{unit}, \code{netout}!0 \\
    \tleveltr{(2)}{P}^\textit{Low} & = & \code{email}?0, \code{emailBox}?\code{true},
    \code{sendButton}?\code{unit}, \code{netout}!1 \\
    \tleveltr{(1)}{P}^\textit{Low,in} & = & \code{email}?0, \code{emailBox}?\code{true},
    \code{sendButton}?\code{unit} \\
    \tleveltr{(2)}{P}^\textit{Low,in} & = & \code{email}?0, \code{emailBox}?\code{true},
    \code{sendButton}?\code{unit} \\
  \end{array}
\end{displaymath}

Finally, we can define a program to satisfy noninterference if, for
every pair of traces such that the inputs at level $S$ are the same,
the outputs at level $S$ are also the same.

To account for generalized lattice levels such as \textit{MaskLower8},
we also need to treat events that are equivalent at a certain level as
the same. For example, at \textit{MaskLower8}, outputs
\code{0xffffffff} and \code{0xffffff00} are the same, since they do
not differ in the upper 32 bits. Thus, we assume for each security
level $S$ there is a appropriate equivalence relation $=_S$, e.g., for
\textit{MaskLower8}, it compares elements ignoring their lower 8
bits. Note that $x =_\textit{Low} y$ is simply $x = y$ and
$x =_\textit{High} y$ is always true.

\begin{definition}[Noninterference]
  \label{defn:noninterference}
  A program $e$ satisfies security policy $P$, if the following formula
  holds:
\begin{displaymath}
  \begin{array}{c}
    \multicolumn{1}{l}{\forall S. \forall t_1, t_2 \in
      \traces(e)\;.} \\
    \tleveltr{\tr_1}{P}^{S,in} =_S \tleveltr{\tr_2}{P}^{S,in}
    \implies \\
    ~~\tleveltr{\tr_1}{P}^S =_S \tleveltr{\tr_2}{P}^S  \\
  \end{array}
\end{displaymath}
\end{definition}

Looking at traces for the insecure app in Fig.~\ref{fig:evilappexample},
we see they violate non-interference, because 
$\tleveltr{(1)}{P}^\textit{Low,in} =
\tleveltr{(2)}{P}^\textit{Low,in}$,
but
$\tleveltr{(1)}{P}^\textit{Low} \neq \tleveltr{(2)}{P}^\textit{Low}$
(they differ in the output).  We note that our definition of
noninterference makes it a 2-hypersafety property \cite{Clarkson:10}.

  
% \subsection{Enforcing policies using symbolic execution}
% \label{sec:symbolic}

% Our semantics for programs (presented in section~\ref{sec:formalism})
% generates sets of symbolic traces, $\tset$.  These traces contain
% symbolic variables on their input channels, some of which must be
% constrained to satisfy the policy.  We now describe how to check a set
% of symbolic traces to see that it satisfies a policy $P$.  From the
% set $\tset$, we form a set of obligations that are sent to an SMT
% solver.  The formulas will mention symbolic inputs and generate
% formulas which will subsequently be checked for satisfiability.  We
% generate our equations so that if the solver returns a satisfiable
% solution, a counterexample can be derived, and if the equations are
% not satisfiable, the program satisfies the policy.

% Programs read inputs along their internal channels, some of which are
% secret (as disignated by the policy $P$).  The extensional policy we
% enforce is to show that, for (subsequences of) any two structurally
% equivalent traces $\tr_1$ and $\tr_2$, if the low observable input is
% the equal, then the sequences of low observable events are the same.
% Because we are using an SMT solver, we instead check if there are any
% traces where we might be able to find a \emph{inequivalent} sequence
% of outputs, given our assumptions about the inputs.  For each pair of
% symbolic traces in $\tset \times \tset$, we assemble the formula
% $test(P,\tr_1, \tr_2')$\footnote{$\tr_2'$ is the trace with the same
%   structure as $\tr_2$, but whose variables have all been replaced by
%   their primed analogs}:

%% \subsection{Policies for our programs in HyperLTL2}

%% \jeff{Not sure what goes in this section}

%% To relate our policies to more traditional information flow policies,
%% we use the language of HyperLTL2.  HyperLTL2 is a logic for stating
%% information flow policies for systems that allows quantification over
%% traces and temporal logic statements about relations between traces.

%% \begin{Definition}[Concretized trace sets]
%%   From a symbolic trace set $\overrightarrow{\tr}$, we can obtain a
%%   (potentially infinite) set of concrete traces,
%%   $\gamma(\overrightarrow{\tr})$.  We define this set as follows: for
%%   all traces $\tr$ containing only concrete values on its channels,
%%   $\tr \in \gamma(\overrightarrow{\tr})$ if and only if the path
%%   condition for $\tr$ is satisfiable when the symbolic variables are
%%   instantiated with the concrete assignments from the corresponding
%%   positions (via its messages) in $\tr$.
%% \end{Definition}

%% \begin{Definition}[Noninterference]
%%   A program $e$ satisfies noninterference with respect to a policy $P$
%%   if:

%%   \forall \tr_1, \tr_2.  \tr_1 \equiv_{L,in} \tr_2 \implies \tr_1
%%   \equiv_{L} \tr_2
%% \end{Definition}

\section{Implementation}
\label{sec:implementation}

Using the ideas from Section~\ref{sec:formalism}, we built a prototype
tool for checking whether Android apps obey GUI-based declassification
policies. We built our tool on top of SymDroid~\cite{Jeon:2012}, a
symbolic executor for Dalvik bytecode, which is the bytecode format to
which Android apps are compiled.

Symbolic execution \cite{King:1976} is simliar to normal program
execution, except a symbolic executor can compute with \emph{symbolic
  expressions} that may contain \emph{symbolic variables} that
represent sets of values. For example, a symbolic executor may execute
an assignment \code{int x=new_sym()}, where the return value of
\code{new_sym} is a fresh symbolic variable representing an arbitrary
32-bit integer.  At conditional branches that depend on symbolic
variables, the symbolic executor invokes an SMT solver, in our case
Z3~\cite{deMoura:2008}, to determine whether one or both branches are
possible. As it follows branches, it extends the current \emph{path
  condition}, which tracks branches taken so far. For example, suppose
the current path condition is $\Phi$, and \code{x} is assigned as
above. Then if the symbolic executor encounters a branch \code{if
  (x>5)...}, it checks whether $\Phi\wedge(x>5)$ is satisfiable. If it
is, it continues execution on the true branch with that as the new
path condition. It also checks whether $\Phi\wedge(x\leq 5)$ is
satisfiable. If it is, it continues execution on the false branch with
that path condition. Notice that both may be satisfiable, in which
case the symbolic executor forks execution, taking both
paths. More details about symbolic execution can be found in the
survey paper by Cadar et al.~\cite{Cadar:13}.

SymDroid uses the features of symbolic execution to implement the
nondeterminism in the \textsc{TInput} rule in our operational
semantics, up to a certain bound. Since we have symbolic variables
available, we also use them to represent arbitrary secret inputs in a
compact way. There are several issues that arise in applying SymDroid
to checking our policies, as we discuss next.

\subsection{Driving App Execution}

Our formal model has built-in notions of callbacks, the message queue,
and non-deterministic inputs. On Android, those features are part of
the framework. We could try to symbolically execute
Android framework code directly, but past experience suggests this is
intractable, since the framework is large, extremely complicated, and
includes native code.

Instead, we created an \emph{executable model}, written in Java, that
mimics key portions of Android needed by our subject apps. Our Android
model includes facilities for generating clicks and
other GUI events (such as the \code{View}, \code{Button}, and
\code{CheckBox} classes, among others). It also includes code for
\code{LocationManager},
\code{TelephonyManager}, and other basic Android classes.

In addition to code modeling the Android framework, the model also
includes simplified versions of Java library functions, such as the
\code{StringBuffer} and \code{StringBuilder} classes.  Our versions of
these APIs implement simplified (unoptimized) versions of methods in
Java and escape to internal SymDroid methods to handle operations that
would be unduly complex to execute symbolically. For instance, SymDroid
represents Java \code{String} objects with OCaml strings, instead of
Java arrays of characters. It then models methods such as \code{String.concat}
with internal calls to OCaml string manipulation functions. Likewise,
reflective methods such as \code{Class.getName} are handled internally.

For each app, we created a driver that uses our Android model to simulate user
input to the GUI. The driver is specific to the app since it depends on the
app's GUI.  The driver begins by calling the app's \code{onCreate} method. It
then calls SymDroid internal methods to set up a map from the app's GUI
elements to channel names. 
% In general, the driver refers to GUI elements
% through their text IDs, which are names used by app developers. Android maps
% these to numeric IDs by using the text IDs as field names in a special class
% called \code{R.id}. The driver uses reflection to retrieve the numeric IDs
% from this class.
%
% Once GUI element IDs are mapped to channel names,
Afterwards, the driver invokes special
methods in the Android model to model GUI events. There is one such method for
each type of GUI element, e.g., buttons, check boxes, etc. Each of these
methods adds an appropriate entry to the trace, then calls an event handler
for the element. The trace entry contains the channel name mapped to this
control's ID, and a value if necessary. The value could indicate, for
instance, which element was selected from a spinner. Event handlers are those
that the app registered through standard Android framework mechanisms,
typically in \code{onCreate}.
%The driver retrieves those ids from public fields in
%the apps, e.g., from \code{BumpApp.sendButton} for the app in
%Fig.~\ref{fig:app-bump}. 
%(All of our subject apps store these ids in
%public fields initialized in \code{onCreate}.)

% It then associates a set of GUI elements in the application with
% channels, so that when clicks are made to a button, a click event
% can be recorded in the trace corresponding to the corresponding
% channel.  Each driver program associates GUI elements with channels:
% for example, in the bump app, the \code{sendBox} button is
% associated with the \code{sendButton} channel, so that whenever that
% button is clicked a \code{unit} value is sent along the
% \code{sendBox} channel.  This will subsequently be used to check the
% policy for the bump app.

Let $m$ be the total number of possible GUI events.  To simulate one
arbitrary GUI event, the driver includes a block that branches $m$
ways on a fresh symbolic variable, performing a different GUI action
in each branch.  Typical Android apps have an unbounded number of
possible input sequences, since they never exit unless the framework
kills them. Thus, we cannot actually explore all possible program
traces. Instead, the user specifies an \emph{input depth}~$n$, and the
driver runs the GUI event block $n$ times (branching on a fresh
symbolic variable each time). In total, then, the driver will execute
$m^n$ paths.

\subsection{Generating Traces}
\label{sec:symbolic-traces}

For each path SymDroid explores, it also maintains a trace of input
and output events, corresponding to the traces from
Section~\ref{sec:formalism}. Internally, the Android model calls various
specially recognized methods to emit events. For example,
\code{Trace.addClick(id)} generates a click event for the given
\code{id}, and similarly for other kinds of events.
%[2/11/15, 5:12:56 PM] Kristopher Micinski: Trace.addToggle(b.getId(), !b.isChecked());
%[2/11/15, 5:13:03 PM] Kristopher Micinski:       Trace.addChoose(spinner.getId(), pos, adapter.getItem(pos));
When SymDroid sees such a call, it adds the event to the current trace
for the current program path.

% When the driver for each program sets up the program state, it
% associates certain GUI elements with channel names by calling into
% SymDroid with a button ID, where it is stored in a map from IDs to
% channel names.  When SymDroid sees the entry of a method corresponding
% to a GUI event (such as a button click or checkbox toggle), it loooks
% up the identifier of the GUI element in the map set up by the driver,
% and appends an event in the trace it records in the path's trace.

In addition to GUI inputs, apps also use secret inputs. We could use
SymDroid to generate concrete secret inputs, but instead we opt to use
a fresh symbolic variable for each secret input. For example, the call
to \code{manager.getEmail} in Fig.~\ref{fig:app-bump} returns a
symbolic variable, and the same for the call to
\code{manager.getPhoneNumber}. This choice makes checking policies
using symbolic execution a bit more powerful, since, e.g., a symbolic
integer variable represents an arbitrary 32-bit integer. In \toolname,
generating a symbolic variable for a secret input also emits a trace
event.

Recall that secret inputs may appear in traces, and thus traces now
may contain symbolic variables. For example, using $\alpha_i$'s as
symbolic variables for the secret email and phone number inputs, the
traces (1) and (2) become
\begin{displaymath}
  \begin{array}{cl}
    \code{email}?\alpha_1, \code{phone}?\alpha_2, \code{emailBox}?\code{true},
    \code{sendButton}?\code{unit}, \code{netout}!\alpha_2 & (1') \\
    \code{email}?\alpha_1, \code{phone}?\alpha_2, \code{emailBox}?\code{true},
    \code{sendButton}?\code{unit}, \code{netout}!\alpha_2 & (2') \\
  \end{array}
\end{displaymath}
%(Note that the concrete traces (1) and (2) can be retrieved from
%these traces by instantiating the $\alpha_i$.)

Note we must take care when using symbolic variables in traces.
Recall that the \textit{level} function checks $t,i \models \phi$ and
then assigns a security level to position $i$. If whether $\phi$ holds
depends on symbolic variables in $t$, then we might not be able to
decide this. For example, if the third element in (1') was
$\code{emailBox}?\alpha_3$, then rather than
$\tlevel{\tr}{P}{0} =\textit{Low}$, we would need to reason with
conditional security levels such as
$\tlevel{\tr}{P}{0} =\textsf{\textbf{if }} \alpha_3 \textsf{\textbf{ then }} \textit{Low}
\textsf{\textbf{ else }} \textit{High}$. We
avoid the need for such reasoning by only using symbolic variables for
secret inputs, and by ensuring the level assigned by a policy does not
depend on the value of a secret input. We leave supporting such
reasoning to future work.

% In our implementation, SymDroid represents all secret inputs
% by symbolic variables, but simulates concrete GUI inputs: this is
% because we require that whether a given declassification condition
% $\phi$ holds at an index $i$ in a (symbolic) trace must be decidable.
% This is not the case in general, because if our GUI inputs
% were symbolic, it would be impossible to decide whether
% $\tlast{\code{emailBox}}{\code{true}}$ held for the above example
% trace for the evil app.  We note, however, that all of our policies
% rely only on the GUI inputs: and these have small domains (compared
% with the secret inputs).  Because of this, we concretize the GUI
% inputs in our implementation, so that the $level$ function is
% definable on symbolic traces.

% To create a secret input, we call a Java method \code{new\_sym}, which
% takes a parameter \code{channel}.  This is intercepted and handled
% within SymDroid, generating a new message on the designated channel.

% SymDroid produces a \emph{finite} set of symbolic traces, which we
% represent with the metavariable $\tset$.

% \begin{figure}
%   \begin{displaymath}
%     \begin{array}{rclp{1in}}
%       \evt & : & \sch ? \{p,\alpha\} \mid \sch ! \{p,\alpha\} &
%       Symbolic event \\
%       t^s & : & \evt ~ \textit{list} , \Phi & Symbolic trace \\
%     \end{array}
%   \end{displaymath}
%   \caption{Symbolic traces.}
%   \label{fig:symbolic-traces}
% \end{figure}

\subsection{Checking Policies with Z3}

Given a set of traces for the program (for inputs sequences up to some
fixed depth), \toolname{} can check noninterference using Z3.

\lstset{language=Caml} 
\begin{figure}
  \begin{lstlisting}
  And ((ReadOnChannel "longitude"), 
          Past(Until(Not(LiteralEvent (`PClick "coarse")),
                       LiteralEvent (`PClick "fine")))), 
  MaskLower8
  \end{lstlisting}
  \caption{Example AST encoding policy for the location toggle
    example.\kris{can't figure out how to center this}}
  \label{fig:ltl-ast}
\end{figure}

SymDroid is implemented in OCaml, and we write our policies as OCaml
lists of declassification conditions, which
are pairs of LTL formulae and security levels. Fig.~\ref{fig:ltl-ast}
gives an example declassification condition for the location toggle
app from Fig.~\ref{fig:app-loc-toggle}. The \code{ReadOnChannel ``longitude''}
construct expresses the $\code{longitude}?\ast$ predicate,
while \code{LiteralEvent} expresses that there was a
click on the designated channel with the specified value.
% The channel
%\code{longitude} comes from the longitude input as a result of
%creating a new secret \code{Location} object, and the \code{coarse} and
%\code{fine} channels are connected to the corresponding buttons in the app via
%the driver.

Given a policy, we implement \emph{level} using standard LTL checking
techniques. We then use Z3 to check whether a
given set of traces satisfies the policy. In
Section~\ref{sec:formalism}, each path produced just a trace, but in
the symbolic executor, paths also include path conditions that may
constraint the symbolic variables in the trace. For example, this
would occur if the code contains a branch on a secret input. Thus,
\toolname{} actually generates a set of trace, path condition
pairs. 

We now describe how to use an SMT solver to check whether a given set
of symbolic traces $\tset$ satisfies the policy $P$. Recall that
Definition~\ref{defn:noninterference} assumes for each $S$ there is an
$=_S$ relation on traces. We use the same relation below, encoding it
as an SMT formula. For our example lattice, $=_\textit{High}$ produces
\code{true}, $=_\textit{Low}$ produces a conjunction of equality tests
among corresponding trace elements, and $=_\textit{MaskLower8}$
produces the conjunction of equality tests of the bitwise and of every
element with 0xffffff00.

Given a trace
$t$, let $t'$ be $t$ with its symbolic variables primed, so that the
symbolic variables of $t$ and $t'$ are disjoint. Given a path
condition $\Phi$, define $\Phi'$ similarly.

\begin{algorithm}
  Let $\tset$ be the set of trace, path condition pairs generated for
  an app, and let $P$ be its security policy. Apply \emph{level}
  across each trace to obtain the level of each event.  For each
  $(t_1, \Phi_1)$ and $(t_2, \Phi_2)$ in $\tset\times\tset$, and for
  each $S$, ask Z3 whether the following formula (the negation of
  Definition~\ref{defn:noninterference}) is unsatisfiable:
  \begin{displaymath}
    \begin{array}{c}
      \tleveltr{\tr_1}{P}^{S,in} =_S \tleveltr{\tr_2'}{P}^{S,in} \land \\
      \tleveltr{\tr_1}{P}^S \neq_S \tleveltr{\tr_2'}{P}^S \land \\
      \Phi_1 \land \Phi_2'
    \end{array}
  \end{displaymath}
  If no such formula is unsatisfiable, then the program satisfies noninterference.
\end{algorithm}
We need to include $\Phi_1$ and $\Phi'_2$ in the formula because they
constrain the symbolic variables in the trace. More precisely, trace
$\tr_1$ represents a set of concrete traces in which its symbolic
variables are instantiated in all possible ways that satisfy $\Phi_1$,
and analogously for $\tr'_2$.

Note that our goal is to prove that the noninterference formula from
Definition~\ref{defn:noninterference} is always true. Hence in the
algorithm we negative it and check satisfiability. If the negation is
satisfiable, then noninterference formula does not hold.
If any formula is satisfiable, then Z3 returns a counterexample, which
Symdroid uses to generate a pair of traces (whose input has been
concretized) as a counterexample.

For example, consider the traces (1') and (2') above, and prime
symbolic variables in (2'). Those traces have the trivial path
condition \sfmt{true}, since neither branches on a symbolic
input. Thus, the formula passed to Z3 will be:

\begin{displaymath}
  \begin{array}{c}
    \alpha_1 = \alpha'_1 \land \code{true} = \code{true} \land \code{()} = \code{()}
    \land \\
    (\alpha_1 \neq \alpha'_1 \vee \code{true} \neq \code{true} \vee
    \code{()} \neq \code{()} \vee \alpha_2 \neq \alpha'_2)
  \end{array}
\end{displaymath}
where the first line is from the equality in the definition and the
second is from the disequality. Thus we can see a satisfying
assignment with $\alpha_1 = \alpha'_1$ and $\alpha_2 \neq \alpha'_2$,
hence noninterference is violated. Had this been the correct
implementation of the app, the last clause of the disjunction would
have been $\alpha_1 \neq \alpha'_1$, and then the formula would have
been unsatisfiable.


% A satisfying assignment is any two values of $v_2$ which are not equal
% (Note that $\alpha_1$ and $\alpha_2'$ are not equated because it is
% assigned level $H$ by the policy).  Note that in the benign variant of
% the app (where $\alpha_1$, the email, is sent), the resulting formula
% would be unsatisfiable, because the output would be $v_1$, resulting
% in the formula:

% \begin{displaymath}
%   \begin{array}{c}
%     \alpha_1 = \alpha_1' \land \code{true} = \code{true} \land () = () \land \alpha_1
%     \neq \alpha_1' 
%   \end{array}
% \end{displaymath}

% This formula has no solutions because of the conflict between the
% equality and inequality for $v_1$ and $v_1'$, induced because the
% $level$ assigned the \code{email} input level $L$.

%% \paragraph*{Generating Traces}

%% For each path SymDroid explores, it also maintains a trace of input
%% and output events, corresponding to the traces from
%% Section~\ref{sec:formalism}.

%% In the formalism, messages are either input or output events.  When
%% the driver for each program sets up the program state, it associates
%% certain GUI elements with channel names by calling into SymDroid with
%% a button ID, where it is stored in a map from IDs to channel names.
%% The driver program explores input events nondeterministically by
%% calling methods that mimic these events and generate the corresponding
%% sequence of method calls in Java.  When SymDroid sees the entry of a
%% method corresponding to a GUI event (such as a button click or
%% checkbox toggle), it loooks up the identifier of the GUI element in
%% the map set up by the driver, and appends an event in the trace it
%% records in the path's trace.  In our case, all secret inputs come from
%% classes in Android's framework (such as the \code{Location} class).
%% To generate secret inputs, we modify our model code to generate
%% symbolic variables for fields of these classes, associating them with
%% a designated channel (e.g., the \sfmt{loc} channel).  Then, whenever
%% one of these objects is created, a secret input is recorded in the
%% path's trace.

% When the app instantiates an object or calls a method, SymDroid checks
% to see whether the model contains an implementation for the relevant
% class or method. If it does, SymDroid uses the model version
% instead. For some methods, such as those that affect the visual state
% of controls, the model contains no implementation; in these cases,
% SymDroid treats the method invocation as a no-op.

% Although applications are linked with Android
% runtime library, some library methods (such as those that call out to
% native code in Android) cannot be implemented purely in Java.  These
% calls are intercepted by SymDroid and implemented in OCaml code to
% accurately model their behavior.  As an optmimization, SymDroid also
% models certain methods (such as those that modify GUI buttons) for
% efficiency reasons.

%Our formalism and policies use traces of messages (and LTL formulas
%about them) to define security for applications.  By contrast, Java
%uses callbacks and variables to designate inputs and constructs
%objects and sends them using APIs (e.g., to interact with the
%network).  

% To model input events, the driver designates a certain set of
% callbacks that can be nondeterministically chosen in a toplevel event
% loop.  The app responds to these callbacks by updating the store,
% installing more callbacks, and sending output messages in a way
% similar to our core formalism in section~\ref{sec:formalism}.  Because
% symbolic execution explores only finite unrollings of potentially
% infinite paths, our symbolic execution is parameterized on the number
% of input events to explore.  SymDroid then executes sets of paths up
% to this event bound, and then uses this path set of in its subsequent
% analysis.  Because each event jumps to a set of $k$ handlers (for the
% number of potential events in a given program), the set of paths grows
% exponentially large in the number of events executed.

\subsection{Minimizing Calls to Z3}

A naive implementation of the noninterference check generates $n^2$
equations (where $n$ is the number of traces produced by the semantics)
to be checked by Z3. However, we observed that many of these equations
correspond to pairs of traces with different sequences of GUI
events. Since GUI events are low inputs in all our policies, these
pairs trivially satisfy noninterference. Thus, we need not send those
equations to Z3 for an (expensive) noninterference check.

We exploit this observation by organizing SymDroid's output traces
into a tree, where each node represents an event, with
the initial state at the root, and traces with common prefixes share the
same ancestor traces in the tree. We systematically traverse this tree
using a cursor $t_1$, starting from the root. When $t_1$ reaches a new
input event, we then traverse the tree using another cursor $t_2$,
also starting from the root. As $t_2$ visits the tree, we do not
invoke Z3 on any traces with fewer input events than $t_1$ (since they
are not low-equivalent to $t_1$). We also skip any subtrees for
which the root input event differs from the corresponding event in
$t_1$---the trace represented by the root is low-inequivalent to $t_1$,
and extending that trace cannot make it low-equivalent to $t_1$.

% To optimize the number of equations sent to the solver, we represent
% traces as trees of events.  Each node in the trace tree is an event
% and a path condition that holds at the time of that event, along with
% pointers to a set of children nodes.  We must associate events with
% their path conditions because we are representing a set of extended
% traces by the parent relation of the tree \kris{worded badly}.

% To run our analysis, we traverse pairs of paths in this tree starting
% at two nodes $t_1$ and $t_2$ (initially both $t_1$ and $t_2$ are the
% root of the trace tree). If one $t_1$ contains the
% n\textsuperscript{th} output event, we look back in the other path
% $t_2$ to find the n\textsuperscript{th} event.  \kris{If the
%   n\textsuperscript{th} event has not occured we wait until it has.}
% We then form the necessary output inequality, along with a set of
% equations asserting low equality along the paths up to that point
% (obtained by following each node up to the root of the tree).  If the
% paths $t_1$ and $t_2$ cannot be low equal (because, for example, they
% contain a different set of low events up to the n\textsuperscript{th}
% output), we stop traversal and prune this path.  This is sound because
% once two paths are low inequivalent (the antecedent of our
% noninterference definition), no extensions will be low equivalent
% (note that events only move down in the lattice).  Otherwise, we
% assert the equation to the SMT solver and check if the last outputs
% could have been different.  If so, we report the counterexample
% generated.  We then recursively traverse each pair of children from
% $t_1$ and $t_2$.

\section{Experiments}
\label{sec:experiments}

\begin{figure*}
\small
\centering
\begin{tabular}{ | l | r | r || r | r | r | r | r | r |}
  \hline
  &&&&& Mem &\multicolumn{3}{c|}{Time (s)} \\ \cline{6-8}
  App & Evts & Nec Evts & Paths & Queries &
  (MB) & Explor. & Analysis & Total \\
  \hline
  Bump & 10 & 3 & 59,049 & 29,514 & 1,848 & 486 & 125 & 612 \\
  Bump (insecure 1) & 10 & 5 & 59,049 & 46,722 & 2,125 & 519 & 212 & 731 \\
  Bump (insecure 2) & 10 & 4 & 59,049 & 24,002 & 2,125 & 506 & 101 & 608 \\
  Location toggle & 7 & 2 & 2,187 & 1,086 & 197 & 14 & 441 & 455 \\
  Location toggle (insecure 1) & 7 & 2  & 2,187 & 127 & 197 & 16 & 22 & 39 \\
  Location toggle (insecure 2) & 7 & 3 & 2,187 & 659 & 197 & 16 & 341 & 357 \\
  Contact picker & 8 & 2 & 65,536 & 5,406 & 4,915 & 832 & 64 & 965 \\
  Contact picker (insecure 1) & 8 & 2 & 236,156 & 34,282 & 17,293 & 5,891 & 475 & 6,364  \\
  Contact picker (insecure 2) & 8 & 2 & 65,536 & 5,406 & 4,915 & 1,270 & 93 & 1,363 \\
  \hline
\end{tabular}
\caption{Results for our test apps}
\label{fig:results}
\end{figure*}

We ran our prototype tool on several subject apps to confirm that our
analysis works in practice.

\subsection{Subject Programs} We ran on three apps, including the two
described in Section~\ref{sec:overview}. We also developed two
insecure variants of each app that violate the
desired security policy in various ways. The apps and their variants
are as follows:

\begin{itemize}
\item \textit{Bump.} The (secure) bump app and its policy appear in
  Fig.~\ref{fig:app-bump}. The first insecure variant counts the number of
  times the send button has been clicked, and sends the value of the
  email after three clicks, regardless of the state of the email
  checkbox. The second swaps the released information---if the email
  box is checked, it releases the phone number, and vice-versa.

\item \textit{Location toggle.} The (secure) location toggle app and
  its policy appear in Fig.~\ref{fig:app-loc-toggle}. The first insecure
  variant always shares fine-grained location information, regardless
  of the radio button setting. The second checks if coarse-grain
  information is selected. If so, it stores the fine-grained location
  (but does not send it yet).  If later the fine-grained radio button
  is selected, it sends the stored location. Recall this is forbidden
  by the app's security policy, which allows the release only of locations
  received while the fine-grained option is set.

\item \textit{Contact picker.} We also developed a contact picker app
  that asks the user to select a contact from a spinner and then
  click a send button to release the contact information over the
  network. The security policy for this app requires that no contact
  information leaks unless it is the last contact selected before the
  button click. (For example, if the user selects contact 1,
  selects contact 2, and then clicks the button, only contact 2 may be
  released.) Note that since an arbitrarily sized list of contacts
  would be difficult for symbolic execution (since then there would be
  an unbounded number of ways to select a contact), we limit the app
  to a fixed set of three contacts.
  The first insecure variant of this app scans the set of contacts for a
  specific one. If found, it sends a message revealing that contact
  exists before sending the actual selected contact. The second insecure
  variant sends a different contact than was selected.
%The second insecure variant selects the subsequent contact in
%the list (rather than the selected contact) and sends it to the
%network.
\end{itemize}

% We model the $i$th contact as $con_i$,
% and the spinner channel as \code{s}:

% \begin{displaymath}
%   \begin{array}{c}
%     \code{con_i}? \land \tfuture \big( \code{send}?() \land
%     last(\code{s})=i \big) \rhd L \\
%   \end{array}
% \end{displaymath}



\subsection{Results}

We ran each subject app and variants 10 times on a MacBook Pro with
8GB ram and a 2.9 GHz Intel i7 processor. We
confirmed that our tool correctly reports the benign app variants as
secure and the insecure app variants as insecure.

Fig.~\ref{fig:results} provides quantitative statics about our
apps.  Recall that our analysis is parametrized by the maximum number
of input events, and the possible number of input sequences
is exponential in this length.  The second column of the figure lists
how many events we used for that app. We used up to 10 events
for each app, or the maximum number of events that would keep the
running time below 20 minutes for the benign variants of the apps.

We verified that the event depths explored for our test apps were
sufficient to discover any violation of the policies.  We did this by
manually generating an automaton for each app where the edges were
messages, and states were an configurations of the app.  We then used
each app's automaton to calculate a minimum number of events required
to generate a counterexample.  These numbers are presented in column
three.

%(note that the first insecure contact picker variant generates a larger
%set of symbolic traces and thererfore takes much longer to analyze).
%For all of our subject apps, the
%number of events exercised were large enough to find all possible
%smallest counterexamples.

The remaining columns summarize data about the runs. The fourth column
lists the number of paths explored by the symbolic
executor. For bump and location toggle, the number of symbolic traces
does not change for the insecure variants. This is because those apps
never branch based on symbolic inputs, and the sets of input sequences
are the same for all variants. On the other hand, the first insecure
contact picker variant generates a larger set of paths, because it
branches on the contact list, a secret input.
The fifth column  lists the number of queries made to the solver. 
The sixth column lists the maximum memory
size. We note that the much larger number of paths in the first insecure
contact picker variant results in much greater memory requirements.

%  We note that maximum memory size does not change accross runs
% for an individual app. \jeff{I don't see that---it looks like it does
%   vary between apps and variants?}
%  This is because SymDroid's memory is dominated
% by exploring feasible symbolic traces.
% The only exception is in the case of the second contact
% picker, which creates a large amount of extra flows and requires a
% much larger heap size to represent the program graph.  

The last three columns show the average running times, broken
down by the exploration phase (where SymDroid generates the set
of symbolic traces) and the analysis phase (where SymDroid forms
equations about this set and checks them using Z3). We see that for bump and contact picker,
the exploration phase dominates the running time, while for the
location toggle app the analysis phase dominates the running time.
Looking into this difference further, we found it is due to amount of
secret data read. Bump and the contact picker both read secret data
just once, while the location toggle app reads secret data multiple
times. Thus, for the latter, some equations passed to the solver
contain many more symbolic variables than for the other apps.

\section{Related Work}
\label{sec:related-work}

Hyperproperties (Clarkson et al., \cite{Clarkson:10}) are statements
about system that allow quantification over multiple executions
simultaneously.  Our definition of noninterference is a 2-hypersafety
property because it quantifies over prefixes of trace pairs.  Recently
HyperLTL\textsubscript{2} (\cite{Clarkson:2014}) has presented a logic
stating temporal properties about trace sets.  We suspect that our
work could be stated as a HyperLTL\textsubscript{2} property, but use
a symbolic executor (rather than a model checker) allowing us to scale
to Android apps.

One core feature of our work is using an SMT solver to check
information flow properties of a program.  This has been addressed in
the work of Rastogi et. al \cite{Rastogi:13} in the context of
optimizing secure multi-party computation (SMC) by ascertaining what
knowledge was necessarily leaked by a computation.  Our work also uses
an SMT solver to check properties about knowledge, but addresses it in
the context of Android applications: which are trace based rather than
batch oriented (as in their framework).

O'Neill et al.~\cite{O'Neill:06} present information flow for
interactive programs using strategies. In our semantics, we do not
assume that the observer has control over the sequence of selections:
our semantics simply assumes that user input is given in a
nondeterministic way.  Thus, we do not consider strategies.  Follow-up
work by Clark and Hunt~\cite{Clark:09} shows that for deterministic
programs a less general approach based on streams suffices.  Our
semantics can be viewed as assuming a total input stream, but we focus
on Android applications and declassification (rather than transition
systems as in that work). The recent work of 
Rafnsson et al.~\cite{Rafnsson:12} focuses on defining security
properties for a wide range of interactive programs 
based on partial input streams.  This is
an orthogonal direction to our work, in the sense that we consider
total streams (i.e., secrets are modeled as total functions on time)
and also programs that are deterministic for a fixed user input
sequence. 

Integrating declassification into usable information flow systems has
a long history in the literature, Sabelfeld and Sands survey this work
in \cite{Sabelfeld:05}.  The most closely related system is SIF
(Servlet Information Flow), which focuses on providing language
features for developing web apps.  SIF is based on Jif \cite{Myers:1999} (an information
flow language variant of Java).  While SIF provides language based
abstraction mechanisms for building web applications, the policies are
checked using the JIF type system, which does not offer an extensional
guarantee about program security.  Vaughan et al~\cite{Vaughan:2011}
define expressive declassification policies for Java: policies that
allow functions of secret information to be released after some
occurs in their semantics.  They define a grammar for events and then
extend the Jif compiler to infer events.  Our policies are focused on
relating declassification to user input, and We leave policy inference as
future work.


% Sabelfeld and Myers~%\cite{Sabelfeld:04}: \emph{delimited release} (*
% MRC: which iirc eventually led to robustness *), which says the
% attacker can't exploit declassification to learn more information than
% is intended.
% Sabelfeld and Sands~%\cite{Sabelfeld:05} present a survey of
% declassification that provided axies and terminology for classifying
% information release.  Our work is most inspired by Li and
% Zdancewic~\cite{Li:05}: \emph{relaxed noninterference}, which
% essentially allows declassification through function application.
% Halpern and O'Neill~\cite{Halpern:08}: semantic characterizations of
% secrecy with epistemic temporal logic.

There is a large body of work that focuses on practical security or
information flow for Android.  Notably, Enck et al.~\cite{Enck:10}
present TaintDroid, a run-time taint mechanism tracking for Android.
TaintDroid, however, tracks only direct flows and does not track
information.
Jia et al.~\cite{Jia:13} presents run-time enforcement of
information flow policies on Android, backed by a process calculus
formalization, but focuses on flows between apps, rather than data
coming from within apps and being sent to the network.
Chen et al. \cite{Chen:13} present Pegasus, a system for inferring
what privileged operations happen as a result of various user inputs.
Pegasus is similar to our work in that it uses a temporal logic to
express conditions under which information can be accessed, but
because Pegasus tracks access control, it cannot speak about how
information is released.  Note that all of our examples require access
to the secret data, but release only a portion of it, and only under
the conditions specified by the policy.  Last, Roesner et
al.~\cite{Roesner:12} define access control gadgets: UI elements that
are used to effect access control policies, and implemented in Android.
 To our knowledge, we are the
first to present and enforce extensional declassification policies and
apply them to Android applications.

\section{Conclusion}
\label{sec:conclusion}

We introduced GUI-based declassification policies, which describe
\emph{what} and \emph{when} information can flow. Policies are defined
using LTL formulae describing event traces, where events include GUI
actions, secret inputs, and network sends. We formalized our policies
using a small core language that models Android's message queue and
event handling. Finally, we described \toolname{}, which uses symbolic
execution to check GUI-based declassification policies on Android, and
showed that \toolname{} correctly enforces policies on on three apps,
with one secure and two insecure variants each.

\section*{Acknowledgments}

This research was supported in part by NSF CNS-1064997 and the
partnership between UMIACS and the Laboratory for Telecommunication
Sciences.

\bibliographystyle{IEEEtran}
\bibliography{paper}

\end{document}
